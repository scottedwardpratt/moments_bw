% !TEX root =  CCmoments.tex

\section{Bose and Fermi Statistics}\label{sec:bose}

Including Bose and Fermi statistics into the recursive relations for partition functions is straight-forward, and was shown in \cite{Cheng:2002jb,Pratt:1999ns}. The method is related to that used for calculating the effects of multi-boson interference for pion interferometry \cite{Pratt:1993uy}. In a fixed volume the partition function can be first treated as the usual procedure of accounting for $n$ identical particles being in different single-particle states. This includes the $1/n!$ term to account for the fact that the particles are indistinguishable, i.e. the Gibbs paradox. If $m_\ell$ indistinguishable particles are in the same single-particle state $\ell$, one must correct the weight by a factor of $m_\ell!$ for each level, which can also be thought of as the analog of the symmetrized relative wave function with all the momenta being equal. For fermions, the weight becomes $(-1)^{\ell-1}m_\ell$. As demonstrated in \cite{Pratt:1999ns}, the recurrence relation to the partition function then becomes
\begin{eqnarray}\label{eq:Zbf}
Z_{A}(Q,B,S)&=\frac{1}{A}\sum_h \sum_n Z_{A-n}(Q-nq_h,B-nb_h,Q-nq_h)z_{h,n}(\pm 1)^{n-1},
\end{eqnarray}
where $z_{h,n}$ is the partition function for $n$ particles in some level, 
\begin{eqnarray}
z_{h,n}&=\sum_\ell e^{-n\beta \epsilon_\ell},
\end{eqnarray}
where $\ell$ refers to single-particle levels of energy $\epsilon_\ell$. The $\pm 1$ refers to bosons or fermions. For hadron gases in the high temperature environments of relativistic heavy ion collisions, only pions have a non-negligible correction from quantum degeneracy. The correction of order $n$ for any level $\ell$ is of the order $e^{-\beta \epsilon_\ell}$ lower than the previous term. This factor is largest for zero momentum, and for pions becomes $e^{-\beta m}$, where $m$ is the pion mass. For the zero-momentum level at a temperature of 150 MeV, the factor is $e^{-m/T}\approx 0.4$, and as the system cools the factor falls slightly \cite{Greiner:1993jn}. For a more characteristic thermal momentum the factor is $\approx 0.1$. For heavier particles the factor is always small in the context of relativistic heavy ion collisions. For example, for a zero-temperature $\rho$ meson the factor is a fraction of a percent.

Given that symmetrization is only being applied to pions, which are bosons, one can incorporate these corrections into the Monte Carlo procedure outlined in Sec. \ref{sec:theoryMC}. For fermions this might be problematic because of the negative weights coming from the $(-1)^{n-1}$ factors in Eq (\ref{eq:Zbf}), but fortunately this is unneccessary because the degeneracy of fermions is negligible in the systems considered here. For pions, the algorithm is adjusted by treating each value of $n$ as being a different species, with charges $nq_h$ and with the partition function calculated with a reduced temperature $T\rightarrow T/n$. If one picks such a species in step 3 of the algorithm, $n$ pions are generated, all with the same momentum. For a finite system, the $n$ pions would be assigned small relative momenta on the order of the inverse system size.

It is well known that bosonic effects can broaden multiplicity distributions, consistent with negative binomial distributions \cite{Carruthers:1983my,Carruthers:1989jj}, making them super-Poissonian. One of the goals of this study is to discern how bosonic statistics alter the skewness and kurtosis. 
