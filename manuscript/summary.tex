% !TEX root =  CCmoments.tex

\section{Summary}\label{sec:summary}
 
The principal goal of this study were to clarify background contributions for higher moments of charge distributions measured by the STAR Collaboration at RHIC, and to state the degree to which current experimental results are either consistent or inconsistent with these contributions. By background, this refers to sources of fluctuations besides those that arise from baryon number or charge clustering due to process such as phase separation. The list of such sources includes charge conservation, Bose corrections, volume fluctuations, and decays of resonances or charge clusters. In order to gain better insight both simple semi-analytic models with a single type of conserved charge, similar to the work performed in \cite{Savchuk:2019xfg}, and a more realistic blast-wave model which includes a more realistic accounting of the STAR acceptance, similar to what was applied in \cite{Oliinychenko:2020cmr}. By using a highly efficient algorithm for Monte Carlo generation of charges in a canonical ensemble, results were produced with small statistical uncertainties and to explore the sensitivity of results to critical parameters of the model.

Of the various background correlations, charge conservation provided the strongest non-Poissonian contributions. Because fourth-order cumulants were defined to subtract contribution from second-order correlations. One might have expected a small contribution. Consistent with the result from \cite{Savchuk:2019xfg} for a single charge, it was found that generating sets of particles from a sub-volume equilibrated according to the canonical ensemble produced values of $C_4/C_2$ which significantly lower than the Poissonian limit of unity. In contrast, the contribution from two-particle decays left did not change the result. This conclusion persisted for the more realistic models incorporating the conservation of all three charges and being filtered by the STAR acceptance. The correlation varied modestly according to the size of the canonical sub-volume for small sub-volumes, due to canonical suppression. A modes sensitivity was also found to the spatial extent of this volume along the beam axis, as the overlay of collective longitudinal flow onto the finite acceptance in rapidity effectively lowers the probability for two balancing charges to both be observed. Due to the fact that baryon density falls with increasing beam energy, the strength of such correlations do depend on beam energy. But this sensitivity was not dramatic. Thus, even though this background contribution is rather large, it is quite smooth with respect to beam energy, so if sharp non-monotonic structures are observed experimentally with respect to beam energy, such structures are not likely to be driven by charge conservation. 

A second source of background arises from multi-particle symmetrization of the outgoing pions. By extending the recursive techniques applied for the canonical ensemble to include symmetrization, it was found that such effects should not affect the skewness or kurtosis unless the pion phase space density were to become surprisingly large. In order for the symmetrization effects to become large, the pion phase space density in the absence of symmetrization would have to double. If this were the case, the pion spectra would be more dramatically altered by Bose effects and the measured HBT radii would have to be significantly altered. 

Volume fluctuations are somewhat of a wildcard of background processes. As shown in Sec. \ref{sec:volumefluc}, such fluctuations can significantly increase the $C_4/C_2$ ratio. However, such processes also increase similar moments of the multiplicity distribution, which counts particles rather than net charges. Thus, it is critical for the experiments to simultaneously present moments of the multiplicity distribution alongside those of the net-charge distributions. This sensitivity was also illustrated with the simple results from a system with one charge and uniform acceptance, restated in Eq. (\ref{eq:savchuk}) from the previous study \cite{Savchuk:2019xfg}. Because volume fluctuations should similarly increase the $C_4/C_2$ ratio for net charge, net strangeness and net baryon number, behavior of such ratios for one type of charge that are not seen in other types of charge can be considered as originating from some other effect. 

Due to the way in which cumulants are constructed, the third and fourth-order cumulants should be impervious to the effects of two-particle decays. However, decays that produce three charged particles manifest themselves in $C_3$ and decays producing four or more charged particles affect $C_4$. Here, it was found that if a significant fraction of charged particles comes from such decays, the higher moments were profoundly altered. However, the decays of such clusters would also affect the multiplicity distribution, which again underscores the importance of the experiments to simultaneously analyze fluctuation of net charge and of the the multiplicity distribution. If such effects were important, it would suggest novel contributions to the dynamics of charge production, outside of the usual paradigm of creating equilibrated distributions of hadrons. 

Finally, the results of the blast-wave calculations were displayed alongside STAR results. The experimental result had much larger statistical errors than the calculations, which limits the conclusions that can be drawn. The fluctuations of net proton were not far from the range of those calculated here. This is consistent with charge conservation being the dominant source of non-Poissonian behavior, i.e. $C_4/C_2\ne 1$ and $C_3/C_1\ne 1$. By no means does this suggest that this is evidence for a lack of more novel sources of correlation in the baryons, such as that arising from phase separation. The experimental uncertainties may be  currently too large to unmask such phenomena. The same holds for the fluctuation of net kaons, which seems to lie above the predictions here, but due to large uncertainties, of the order of 50\%, it is impossible to make any strong conclusions.

It is the observed moments of the net-charge distribution that are most perplexing. Even with the large statistical uncertainties, of several tens of percent, it is clear they are nearly a factor of two higher than expectations of the blast-wave calculations. Given that the net-proton distributions are roughly in line with the model predictions, this discrepancy cannot be explained by volume fluctuations, or equivalently by the systematics of event binning. Given that phase transition phenomena are expected to manifest themselves mainly in the net-proton distributions, this would suggest that the decays of larger clusters into four or more charged particles might be present. If such processes were present, it would motivate a rethinking of models of chemical evolution and charge production in heavy-ion collisions. In order to confirm or dismiss this hypothesis, it is imperative that a simultaneous analysis of multiplicity distributions be undertaken from the same data sets with the same cuts on centrality. Given the greatly improved data sets currently being analyzed by STAR in the Beam Energy Scan II program at RHIC \cite{Yang:2017llt}, this puzzle should be clarified in the next year or two.

Finally, the studies here help point the way to future improvements in modeling. The picture of independent canonical sub-volumes is crude. It does incorporate the truth that charge conservation is enforced locally, over some length scale, and is sufficient to provide the understanding of how large such effects might be. However, in reality baryon number, electric charge and strangeness are created and evolve in different ways. Strangeness tends to be created early in the collision, thus allowing the balancing strange and anti-strange quarks to separate before the emission of the hadrons to which they are asymptotically assigned. In contrast, up and down quarks are more likely to be produced later in the reaction. Thus, the characteristic canonical volumes should have different sizes and different longitudinal extents depending on whether one is considering up, down or strange quarks. This is also true for off-diagonal correlations, e.g. correlations between strange and up. These only appear in the hadronic phase. Fortunately, a diagrammatic formalism has been developed for evolving such two- three- and four-body correlations as a function of the positions of each charge. These equations are based on knowing the chemical evolution of the system and the diffusion constants for each type of charge. Such calculations have been performed for two-body correlations and compared to experimentally measured charge-balance functions \cite{Pratt:2018ebf,Pratt:2019pnd} or fluctuations \cite{Aziz:2004qu}. Unfortunately, the method for three- and four-body correlations  is challenging to implement numerically \cite{Pratt:2019fbj}. It is tractable, but would require significant effort. The study presented here suggests that such calculations would be warranted if one wanted to reproduce the moments of these distributions for each type of charge, and especially if one wants to consider cross terms \cite{Abdelwahab:2014yha}, such as moments involving powers of both charge and net baryon number. Otherwise, given that charge conservation effects are expected to evolve smoothly with beam energy, one could simply see whether measurements of the ongoing Beam Energy Scan at RHIC unveil any sharp features, and assign these features to more novel types of physics. 