\documentclass[aps,prc,twocolumn,nofootinbib,showpacs,superscriptaddress,groupedaddress]{revtex4-1}
\usepackage{amsmath,amssymb,amsbsy,bm}
\usepackage{graphicx}
\usepackage{comment}
\usepackage{float}
\usepackage[colorlinks=true,linkcolor=blue,citecolor=blue,urlcolor=blue]{hyperref}
\newcommand\eqnumber{\addtocounter{equation}{1}\tag{\theequation}}

\begin{document}

\title{Charge Conservation and Higher Moments of Charge Fluctuations}
\author{Scott Pratt}
\affiliation{Department of Physics and Astronomy and National Superconducting Cyclotron Laboratory\\
Michigan State University, East Lansing, MI 48824~~USA}
\author{Rachel Steinhorst}
\affiliation{Department of Physics and Astronomy\\
Michigan State University, East Lansing, MI 48824~~USA}
\date{\today}

\pacs{}

\begin{abstract}

\end{abstract}

\maketitle

\section{Introduction}\label{sec:intro}

The fluctuation of conserved charges is a standard means by which to investigate and gauge phase transitions. At the critical point correlation lengths diverge, which results in peaks in charge fluctuations as one approaches the critical point. For systems with first-order phase transitions, fluctuations become infinite whenever one is in a region of phase separation, as the liquid and gas regions prefer to become infinite in extent when the surface energy is non-zero. The growth of fluctuations becomes increasing dramatic as one considers progressively higher-order fluctuations. In a volume $V$, fluctuations of a charge $Q$ can be defined as
\begin{eqnarray}\label{eq:kappadef}
\mathcal{M}_N&\equiv&\frac{1}{V}\langle(Q-\bar{Q})^N\rangle=\frac{1}{V}\sum_n P_n(n-\bar{n})^N,
\end{eqnarray}
when particles have unit charge. The measure is increasingly sensitive to the tails of the multiplicity distribution, $P_n$, as $N$ increases. The free energy, $F\sim a_2(n-\bar{n})^2 + a_3(n-\bar{n})^3\cdots$, is minimized for $n=\bar{n}$, but the quadratic part vanishes at the critical point, $a_2\rightarrow 0$, which allow the fluctuations to grow and be dominated by higher-order terms which manifest themselves in the tails of the distribution. 

The properties of the QCD  transition, deconfinement and the restoration of chiral symmetry, are not well understood at finite baryon density. There exists the possibility that this transition is a true phase transition, with a critical point at several times nuclear density and with a critical temperature close to the pion mass. If this is the case, it begs the question as to whether the conditions for phase separation or for critical phenomena can be reproduced in the laboratory. Heavy ion collisions at high energy, measured at the Relativistic Heavy Ion Collider (RHIC) or a the LHC, can produce mesoscopic regions a temperatures of a few hundred MeV, which is well above the expectations for a critical temperature, and densities of several times nuclear matter density. 

High-energy heavy ion collisions are characterized by strong explosive collective flow. Measurement is confined to the outgoing asymptotic momenta, but because of strong flow, correlations in coordinate space manifest themselves as correlations in relative momentum. Thus, measurements of correlations binned by relative momentum or charge fluctuations within some defined region of momentum space serve as surrogates for the corresponding observables in coordinate space. Indeed, measurements of charge and baryon number fluctuations have been performed at RHIC. By adjusting the beam energy of the colliding nuclei, experiments at RHIC have explored conditions at which novel phase phenomena might occur. Fluctuations of electric charge and baryon number have been especially popular. An initial scan of beam energies was rather inconclusive, but measurements with greatly improved statistics are currently being analyzed. 

In addition to the finite system size (event multiplicities might number in the thousands), the novel states of matter created in heavy-ion collisions persist for $\lesssim 10$ fm/$c$. This severely limits the degree to which phases can separate or to which critical fluctuations can grow. From the initial overlap of the colliding nuclei, until the mean time at which emitted particles experience their last collision, the collision lasts a few tens of fm/$c$. This limits the degree to which conserved charges can separate from one another. For example if a strange and an anti-strange quanta are produced together, their separation is limited by diffusion, and the enhancement of seeing a second charge of opposite sign having seen a first charge are known as charge-balance correlations.

The goal of this paper is to gauge the degree to which charge-balance correlations affect higher-order correlations. Such correlations need to be understood as potential sources of background before making firm arguments to have observed phenomena related to phase transitions. It is well known that charge-balance correlations are readily measurable at the two particle level, $N=2$ in Eq. (\ref{eq:kappadef}), and that they explain the bulk of the $N=2$ fluctuation measurement. However, their impact on $N=3,4$ fluctuations has not be investigated in great detail. For instance, the four-particle measure of correlation,
\begin{eqnarray}
C_4&=&\mathcal{M}_4-3\mathcal{M}_2^2,
\end{eqnarray}
which is based on cumulants, subtracts much of the contribution to the $\mathcal{M}_4$ coming from purely two-particle correlations. However, charge conservation can involve multiple particles, and it is not clear the degree to which a cumulant-based measure, like the kurtosis, is affected by charge conservation. A simple model, where charge conservation is invoked through sampling a canonical ensemble, is presented in the next section. Although the method is different, this is similar to ideas that have been pursued recently \cite{Oliinychenko:2020cmr}. But with the method used here, events can be generated in a more efficient algorithm, which then enables the more systematic set of studies presented here.

After a brief review of cumulants and the definitions of skewness and kurtosis in Sec. \ref{sec:cumulaants}, the method for exact solution of the canonical ensemble describing a multi-component, multi-charge hadron gas is presented in Sec. \ref{sec:theory}. The method includes all species and can account for Bose or Fermi statistics, but neglects interaction between hadrons. These techniques extend those used for canonical ensembles used to study isospin fluctuations of a hadron gas \cite{}, nuclear fragmentation \cite{}, the level density of a Fermi gas, and the effect of restricting a quark-gluon plasma (QGP) to having fixed charge, including being in an overall color singlet \cite{}. Exact methods for calculating correlations up to fourth-order are presented. Unfortunately, to include realistic acceptance effects and complex decays, the exact expressions are no longer viable. However, as shown in Sec. \ref{sec:theoryMC}, the exact expressions show how sample events can be generated. An event, defined by a set of particles and momenta, are generated with perfect independence from one another, and perfectly reproduce the canonical expressions from Sec. \ref{sec:theoryexact}. 

Conservation of charge is modeled by imagining independent sub-volume ``patches'' which individually conserve charge. The patch volume is set by how far charges have separated from another by the time chemical freezeout occurs, just after hadronization. After this time, the volumes may grow and overlap, but the chemical makeup of the particles originating from each patch is assumed to be fixed. Sections \ref{sec:ideal1} and \ref{sec:ideal3} shows how fluctuation observables from charge conservation depend of patch size, measurement efficiency, and the chemical decoupling temperature by investigating systems with fixed efficiency. A very simplified system with a single charge, as already investigated in detail in \cite{}, is reviewed in Sec. \ref{sec:ideal1}, while the full hadron gas, with three conserved charges, is presented in Sec. \ref{sec:ideal3}. The importance of Bose corrections for pions is also studied. Charge from the initial nuclei existed before the collision and might randomly populate the entire collision volume, whereas charge created from the breakup of color flux tubes, hadronization and decays, should be conserved locally. Thus, the effect of random placing some of the initial charge across the entire collision volume is also investigated. A more realistic picture of the acceptance and efficiency is presented in Sec. \ref{sec:blast}. A blast-wave model, which is a thermal model superimposed onto a simplified model of collective flow, is combined with a sampling of a canonical ensemble of a given patch. The resulting particles are then analyzed using a simulation of the acceptance and efficiency of the STAR Collaboration at RHIC. Results are compared to measurements at STAR to discern whether charge conservation effects might be responsible for a significant part of the observed correlation. Results are then summarized in Sec. \ref{sec:summary}.

\section{Cumulants, Skewness and Kurtosis}

To keep the more manuscript self-contained, a brief review of cumulants and the definition of skewness and kurtosis is presented here.

Cumulants of a charge distribution are defined by 
\begin{eqnarray}
C_1&=\langle Q\rangle,\\
C_2&=\langle(Q-\bar{Q})^2\rangle,\\
C_3&=\langle(Q-\bar{Q})^3\rangle,\\
C_4&=\langle(Q-\bar{Q})^4\rangle-C_2^2,
\end{eqnarray}
where $Q$ is the net charge. Here, $Q$ might refer to baryon number, to strangeness, or to the electric charge measured in units of $e$.

Rather than showing the cumulants $C_n$, ratios are presented to help minimize trivial dependences on system size. The skewness, $S$, is a measure of the third moment,
\begin{align*}\eqnumber
S&=\frac{C_4}{C_2^{3/2}}.
\end{align*}
This definition has the advantage in being dimensionless, but it does not become independent of volume in the limit of large volumes. Thus, it is more common to consider the ratio
\begin{align*}\eqnumber
S\sigma&=S\sqrt{C_2}=\frac{C_3}{C_2},
\end{align*}
which becomes an intensive measure in the limit of larger volumes. The kurtosis is a measure of four-particle correlations,
\begin{align*}\eqnumber
K&=\frac{C_4}{C_2^2},
\end{align*}
but instead of $K$, one typically chooses
\begin{eqnarray}
K\sigma^2&=\frac{C_4}{C_2},
\end{eqnarray}
to find an intensive measure of the fluctuation, at least for large volumes. 

The measures $S\sigma$ and $K\sigma^2$ approach simple values in the limit that the distributions would be Poissonian. For Poissonian emission the observation of a charge in one region of momentum space is uncorrelated with the emission into any other space. Thus, particles are correlated only with themselves. If charges appear only in integral positive units, one can apply the usual expression for the Poissonian moments where the mean is $\eta$, 
\begin{align*}\eqnumber
C_1&=\bar{n}=\eta\\
C_2&=\langle(n-\bar{n})^2\rangle=\eta,\\
C_3&=\langle(n-\bar{n})^3\rangle=\eta=C_1,\\
C_4&=\langle(n-\bar{n})^4\rangle-3C_2^2=\eta.
\end{align*}
If there exist both positive and negative charges, the distribution of the net charge can be derived by convoluting the two distributions. Convoluting two Poissonians results in a Skellam distribution. If the mean number of positives is $\eta_+$ and the mean number of negatives is $\eta_-$, the distribution of net charge for a Skellam distribution, $Q=n_+-n_-$, yields the following cumulants
\begin{align*}\eqnumber
C_1&=\eta_+-\eta_-,\\
C_2&=\eta_++\eta_-,\\
C_3&=\eta_+-\eta_-=C_1,\\
C_4&=\eta_++\eta_-=C_2.
\end{align*}
Thus, if charges are produced in an uncorrelated fashion in increments of either $\pm 1$, the skewness and kurtosis become
\begin{align*}\eqnumber
S\sigma&=\frac{C_3}{\sigma^2}=\frac{C_1}{\sigma^2},\\
K\sigma^2&=\frac{C_4}{\sigma^2}=1,
\end{align*}
where $\sigma^2\equiv\langle(Q-\bar{Q})^2\rangle=\eta_++\eta_-$. It might have made more sense to consider $C_3/C_1$ rather than $S\sigma$, as that ratio would be unity in the limit that emission is uncorrelated. However, because experimental analyses have often concentrated on the two ratios above, higher order correlations in this paper will also be presented in terms of these ratios.

Moments depend on the efficiency $\alpha$ with which particles are measured. In the limit of vanishing efficiency all distributions of positives or of negatives tend to become Poissonian, and the distribution of the net charge will thus become a Skellam.  This can be understood by having the probability of observing a positive charge being $\alpha\rightarrow 0$ and the probability of observing two charges being proportional to $\alpha^2$ being negligible. This assumption would fall through if all positive charges were measured in pairs, i.e. multiple charges on each measured particle, but for final-state hadrons the charges are only $\pm 1$. As $\alpha\rightarrow 0$, the moments are dominated by the probability of observing either zero charges or a single charge, and the moments for counting particles of a given charge are
\begin{eqnarray}
\langle n^m\rangle&=\alpha,
\end{eqnarray}
for all $m>0$. It is then easy to see that $\langle (n-\bar{n})^m\rangle=\alpha$, which quickly leads to the result that as $K\sigma^2=C_3/C_1=1$.

If the net charge is fixed, a non-perfect efficiency is required to produce fluctuations. For fixed charge, the efficiency divides into two sets, the measured and non-measured. Each set fluctuates equally, but oppositely, relative to the mean. Thus, all the even moments of net charge will have an even reflection symmetry about an efficiency of $1/2$, and the odd moments will have an odd symmetry,
\begin{eqnarray}
\label{eq:alphasymm}
C_n(1/2+\delta\alpha)&=\left\{\begin{array}{rl}
C_n(1/2-\delta\alpha),&n=2,4,6\cdots\\
-C_n(1/2-\delta\alpha),&n=3,5,7\cdots~.\end{array}\right.
\end{eqnarray}

\section{Recursive Techniques for Generating Canonical Partition Functions}\label{sec:theoryexact}

For non-interacting particles the canonical partition function can be calculated exactly, or at least to the level that all partitions of $A\le A_{\rm max}$ hadrons are taken into account, with the exact solution being reached with $A_{\rm max}=\infty$. For our case, we conserve three quantities: the electric charge $Q$, the baryon number $B$ and the strangeness $S$. For states $i$ with energies $E_i$, the partition function,
\begin{eqnarray}
Z(Q,B,S)&=&\sum_{i,Q_i=Q,B_i=B,S_i=S}e^{-\beta E_i},
\end{eqnarray}
where $Q_i,B_i$ and $S_i$ are the discrete values of the conserved quantities for the state $i$, can be calculated recursively. The function $Z_A(Q,B,S)$ refers to the subset of states with $A$ hadrons,
\begin{eqnarray}
Z(Q,B,S)&=&\sum_{A\ge 0}Z_A(Q,B,S).
\end{eqnarray}
The recursive procedure begins with
\begin{eqnarray}
Z_{A=0}(0,0,0)&=&1,
\end{eqnarray}
is the canonical partition function of the vacuum. The contribution for a given $A$, $Z_A(Q,B,S)$, can be written as 
\begin{eqnarray}\label{eq:recurrence}
Z_A(Q,B,S)&=&=\frac{1}{A}\sum_h z_hZ_{A-1}(Q-q_h,B-b_h,S-s_h),
\end{eqnarray}
where $z_h$ is the single-particle partition function for hadron species $h$, which has charges $q_h,b_h$ and $s_h$. This was proved in \cite{Pratt:1999ht}, and can be understood by realizing that one can count all the ways to arrange $A$ hadrons with a given charge by considering all the ways to arrange one hadron multiplied by all the ways to arrange the remaining hadrons. To avoid double counting, a factor of $1/A$ is applied. For a fixed charge the probability to have $A$ hadrons is,
\begin{align*}\eqnumber
P(A)&=&\frac{Z_A(Q,B,S)}{\sum_A Z_A(Q,B,S)}=\frac{Z_A(Q,B,S)}{Z(Q,B,S)}.
\end{align*}
In practice, the sum over $A$ is cut off at some $A_{\rm max}$, but in our studies here that cutoff is made large enough that contributions to $Z$ for $A>A_{\rm max}$ are negligible. Thus, once builds the partition function from $A=0$ to $A_{\rm max}$ one has the partition function for all $Q,B,S$. 

Once the partition function is calculated one can also calculate the multiplicities and moments of observing specific species. For example, the multiplicity of species $h$ in a system with charge $Q,B,S$ is
\begin{align*}\eqnumber
\langle N_h\rangle &= z_h\frac{Z(Q-q_h,B-b_h,S-s_h)}{Z(Q,B,S)}.
\end{align*}
This also provides expressions for the various charges, e.g.,
\begin{align*}\eqnumber
\langle Q\rangle &= \sum_h q_hz_h\frac{Z(Q-q_h,B-b_h,S-S_h)}{Z(Q,B,S)}.
\end{align*}
Spectra can also be calculated
\begin{align*}\eqnumber
\frac{dN_h}{d^3p}&=\frac{(2s_h+1)\Omega}{(2\pi\hbar)^3}e^{-E_h(p)}\frac{Z(Q-q_h,B-b_h,S-S_h)}{Z(Q,B,S)}.
\end{align*}
Higher moments can also be extracted,
\begin{align*}\eqnumber
\langle N_hN_{h'}\rangle &= \delta_{hh'}z_h\frac{Z(Q-q_h,B-b_h,S-s_h)}{Z(Q,B,S)}\\
&\hspace*{-30pt}+z_hz_{h'}\frac{Z(Q-q_h-q_{h'},B-b_h-b_{h'},S-s_h-s_{h'})}{Z(Q,B,S)}.
\end{align*}
It is straightforward to extend this expression to the fluctuation of charges.

These expressions can also be extended to consider non-additive conservation laws. Net isospin conservation of a hadron gas was invoked in \cite{Cheng:2002jb}, i.e. restricting the states to being in an iso-singlet.  For quark-gluon states were restricted to being in both an iso-singlet and a color singlet in \cite{Pratt:2003jd}. For Bose corrections,
\begin{align*}\eqnumber\label{eq:bose}
Z_A(B,Q,S)&=\frac{1}{A}\sum_{h,n}z_{h,n}Z_{A-1}(Q-nq_h,B-nb_h,S-ns_h),\\
z_{h,n}&=(2s+1)\sum_p e^{-nE_p/T}.
\end{align*}
For the conditions of a hadron gas in high-energy heavy-ion collisions only pions are significantly affected by quantum degeneracy. 




\section{Generation of Uncorrelated Sample Events}\label{sec:theoryMC}

Complicated experimental acceptances are difficult to incorporate into expressions for the moments. It is then easiest to generate entire events via Monte Carlo, and filter the events through the acceptance. The Monte Carlo procedure involves choosing a hadron proportional the number of ways the system might have such hadron, i.e. a product of the partition function of the individual hadron multiplied by the partition function of the remainder. The procedure becomes:
\begin{enumerate}
\item Calculate and store the partition function, $Z_A(Q,B,S)$, up to some size $A\le A_{\rm max}$ for all $Q,B,S$ that might ultimately couple back to a given $A=A_{\rm max}$ for the given total values $Q,B,S$. 
\item For total charge $Q,B,S$, choose the number of hadrons $A$ proportional to $Z_A(Q,B,S)/Z_A(Q,B,S)$.
\item Choose a hadron $h$ proportional to the probability $z_hZ_{A-1}(Q-q_h,B-b_h,S-s_h)/Z_A(Q,B,S)$. If Bose degeneracy is to be taken into account this procedure is slight modified as described in Sec. \ref{sec:bose}.
\item Choose the momentum proportional to the thermal weight $e^{-E_p/T}$.
\item Repeat (3,4) but with $A,Q,B,S$ being replaced by $A-1,Q-q_h,B-b-h,S-s_h$. The procedure is finished when $A=0$.
\end{enumerate}
Bose effects for pions can be included by altering the second and third steps above. In addition to choosing a hadron, one might also consider adding a $2-$pion, $3-$pion or $n-$pion state. Simply treat the $n-$pion state as if it were a resonance with charges $nq_h,nq_b,nq_s$, but with a partition function $z_{h,n}$ described above. If one chooses to make such a ``particle'', the $n$ identical particles are are given the same momentum and spin projection, using a Boltzmann distribution with a temperature $T/n$. To be more realistic the momenta should be smeared by some momenta $\approx\hbar/R$, where $R$ is a characteristic size of the system. If Fermi effects were to be included, one could calculate the partition function by adding a factor $(-1)^{n-1}$ to each term in the sum described in Eq. (\ref{eq:bose}). However, because some of the weights are negative, Monte Carlo procedures can become problematic. Fortunately, for the systems considered here are extremely hot, and degeneracy of fermions is negligible.

Storing the partition function can require substantial memory for larger $A_{\rm max}$ because the indices $Q,B$ and $S$ must also vary over a range of order $\pm A_{\rm max}$, so memory usage roughly scales with $A_{\rm max}^4$. Because one is usually interested in calculations with total charge near zero, one can ignore partition functions for charges that cannot couple back to the fixed overall charge at $A_{\rm max}$. Once $A$ exceeds $A_{\rm max}/2$, the calculations here cutoff values of $Q,B$ and $S$ that could not ultimately affect the $Q=B=S=0$ partition function for $A=A_{\rm max}$. Even with this savings, partition functions with $A_{\rm max}=250$ could require approximately 13 GB of memory, and need on the order of 10 minutes to calculate on a single processor. For $A_{\rm max}=125$, less than a GB of memory was needed and partition functions could be calculated in less than a minute. For hadron gases at temperatures of 150 MeV, $A_{\rm max}=250$ was readily sufficient for patch volumes $\lesssim 700$ fm$^3$. If multiple patch volumes are to be explored for the same temperature, computational time can also be saved by realizing that the partition functions scale as $\Omega^A$. Thus, if one performs a calculation for some initial volume $\Omega_0$, scaling can provide results for new volumes with minimal computation.

Once the partition function is calculated, event generation is remarkably fast. The time to generate an even scales linearly with the volume, or equivalently, linear with the average number of particles generated. Running sufficient events to generate a million individual particles can be accomplished within a few seconds on a single CPU. Unlike Metropolis methods where events are modified by considering small changes to existing events, such as in \cite{}, each event in this is independent as long as the random numbers are without correlation.

\section{Bose and Fermi Statistics}\label{sec:theorybosefermi}

Including Bose and Fermi statistics in calculating the partition functions is straight-forward, and was shown in \cite{}. The method is related to that used for calculating the effects of multi-boson interference for pion interferometry. In a fixed volume the partition function can be first treated as the usual procedure of accounting for $n$ identical particles being in different single-particle states. This includes the $1/n!$ term to account for the fact that particles are indistinguishable, i.e. the Gibbs paradox. If $m_\ell$ indestinguishable particles are in the same single-particle state $\ell$, one must correct the weight by a factor of $m_\ell!$ for each level, which can also be thought of as the analog of the symmetrized relative wave function with all the momenta being equal. For fermions, the weight becomes $(-1)^{\ell-1}m_\ell$. As demonstrated in \cite{}, the recurrence relation to the partition function then becomes
\begin{align*}\eqnumber\label{eq:Zbf}
Z_{A}(Q,B,S)&=\frac{1}{A}\sum_h \sum_n Z_{A-n}(Q-nq_h,B-nb_h,Q-nq_h)z_{h,n}(\pm 1)^{n-1},
\end{align*}
where $z_{h,n}$ is the partition function for $n$ particles in any level, 
\begin{align*}\eqnumber
z_{h,n}&=\sum_\ell e^{-n\beta(\epsilon_\ell-\mu_iq_i)}.
\end{align*}
The $\pm 1$ refers to bosons or fermions. For hadron gases in the high temperature environments of relativistic heavy ion collisions, only pions have a non-negligible correction from quantum degeneracy. The correction for order $n$ for any level $\ell$ is of the order $e^{-\beta(\epsilon_\ell-\mu_iq_i)}$ lower than the previous term. This factor is largest for zero momentum, and for pions becomes $e^{-\beta(m-\mu)}$. The chemical potential is $\mu_Q q_i$ if the pions are chemically equilibrated, but if the system is not chemically equilibrated $\mu$ is adjusted to fit the average pion density. For the zero-momentum level at a temperature of 150 MeV, the factor is $e^{-m/T}\approx 0.4$, and as the system cools probably falls slightly \cite{gong}. For a more characteristic thermal momentum the factor is $\approx 0.1$. For heavier particles the factor is always small. For example, for a zero-temperature $\rho$ meson the factor is a fraction of a percent.

Given that symmetrization is only being applied to pions, which are bosons, one can incorporate these corrections into the Monte Carlo procedure outlined in Sec. \ref{sec:theoryMC}. For fermions this might be problematic because of the negative weights coming from the $(-1)^(n-1)$ factors in Eq (\ref{eq:Zbf}). Here, the algorithm is adjusted by treating each value of $n$ as being a different species, with charges $nq_h$ and with the partition function calculated with a reduced temperature $T\rightarrow T/n$. If one picks such a species in step 3 of the algorithm, $n$ pions are generated, all with the same momentum. For a finite system, the pions would have a small relative momentum or order the inverse system size.

It is well known that bosonic effects can broaden multiplicity distributions \cite{negativebinomial}, making them super-Poissonian. One of the goals of this study is to discern how bosonic statistics alter the kurtosis or skewness. 

\section{Single-Charge Model with Uniform Measurement Efficiency}

For context, we first review an extremely simplified model of a system with only one conserved charge, and where the species all have charge of either $-1,0$ or $+1$. Further, Bose and Fermi degeneracy is ignored and a grossly simplified version of the acceptance is assumed. The chance that any particle is observed is a fixed probability $\alpha$, with the acceptance probabilities of multiple particles being uncorrelated. Many of the expressions shown here were already derived in \cite{Savchuk:2019xfg}. In addition to the results from the canonical ensemble, a system of uncorrelated neutral particles that decays to charged particles is also presented to illustrate the difference in charge conservation manifests itself in fluctuation observables depending on whether the charges arise directly from an equilibrated system, or whether they arise from the decays of uncorrelated neutral particles.

As before, the partition functions are labeled both the charge $Q$ and by the net number of hadrons, in this case charged hadrons. Because particles have charges only of $\pm 1$, the partition function for $z=Z_{A=1,Q=1}=Z_{A=1,Q=-1}$ forms the basis of all other partition functions. From the methods of the previous section
\begin{align*}\eqnumber
Z_{A,Q}&=\frac{z}{A}\left[Z_{A-1,Q-1}+Z_{A-1,Q+1}\right].
\end{align*}
This relation rapidly reproduces the partition function for all $A$ and $Q$ for $A$ of the order of a few hundred.

As an aside we point out that the partition functions from this method agree perfectly with those generated by considering equilibrated reaction rates. For a given charge $Q$, if $Q$ is even only even values of $A$ are allowed, and if $Q$ is odd, only odd values of $A$ are possible. If there are reactions that annihilate hadrons, proportional to the density of positives multiplied by the density of negatives, and thermal rate at which pairs are created, which is depends only on temperature, then at equilibrium,
\begin{equation}
\alpha \frac{(A-Q)(A+Q)}{4}Z_{A,Q}=\beta Z_{A-2,Q}.
\end{equation}
Here, annihilation is described by the constant $\alpha$ which depends on the product of the number of positives, $(A+Q)/2$, and the number of negatives $(A-Q)/2$. Beginning with
\begin{equation}
Z_{A,Q=\pm A}=\frac{z^A}{A!},
\end{equation}
which gives the $Z_{A,Q}$ for the lowest possible $A$ given a specific $Q$, one can generate the partition function for all higher values of $A$ using Eq. (\ref{eq:reacrate}). If the ratio $\beta/\alpha$ is chosen to be $z$,
\begin{equation}
Z_{A,Q}=\left\{\begin{array}{rl}
\frac{z^A}{[(A-Q)/2]![(A+Q)/2]!},&A-Q~{\rm is~even},\\
0,&A-Q~{\rm is~odd}.\end{array}\right. 
\end{equation}
This expression satisfies the recurrence relation from the previous section, Eq. (\ref{eq:recurrence}).  The probability of having $A$ charged particles for fixed charge $Q$ becomes,
\begin{eqnarray}\label{eq:PAgivenQcanonical}
P(A|Q)&=&\frac{Z_{A,Q}}{\sum_A Z_{A,Q}},\\
&=& \frac{z^A}{[(A/2)!]^2}\left\{\sum_A \frac{z^A}{[(A/2)!]^2}\right\}^{-1},~Q=0.
\end{eqnarray}

Decays produce equal amounts of positive and negative charge, but if the decays occur after chemical freezeout, there is no balancing back reaction. This leads to a qualitatively different multiplicity distribution than that of the equilibrated system of charges described in Eq. (\ref{eq:PAgivenQcanonical}). For example if one has uncorrelated emission of neutral particles that decay to form charged pairs, the probability of having $A_0$ decaying particles is Poissonian, and the probability of having $A$ final particles is
\begin{eqnarray}
P(A|Q=0)&=&P_0(A/2)=\frac{\eta^n}{(A/2)!}e^{-\bar{A}/2},~~~A=0,2,4\cdots.
\end{eqnarray}


\section{Hadron Gas with Uniform Efficiency}\label{sec:ideal}

Canonical partition functions depend on the temperature $T$, the volume $V$, the conserved charges $\vec{Q}=(Q,B,S)$, and whether or not Bose corrections are included. Of course, charges don't fluctuate in a canonical ensemble if all the particles are measured, so to understand moments of the charge distribution, acceptance and efficiency must be included. Thus, in addition to $T,V$ and $\vec{Q}$, one needs to understand how the acceptance and efficiency change the result. In this section, to better understand how fluctuations are affected by the parameters above, a grossly simplified acceptance and efficiency is employed. Each particle has a probability $\alpha$ of being observed, regardless of its momentum. 

First, we set the net charges to zero, and study the effects of changing $T,V$ and $\alpha$. Ensembles of particles are generated according to the Monte Carlo procedure described in Sec. \ref{sec:theory} according to $T$ and $V$. Unstable particles are then decayed with the relative momenta of the particles determined by the invariant mass of the decaying particles. For these calculations all unstable resonances are produced according to their pole mass. Each particle is then randomly kept or ignored with probability $\alpha$. 

The dependence on the efficiency $\alpha$ is illustrated in Fig. \ref{fig:eff_T150V100B0}. The volume was fixed at $V=100$ fm$^3$, the temperature was set to $T=150$ MeV, with all the charges set to zero. The results have a mirror symmetry about $\alpha=0.5$ as expected from the discussion surrounding Eq. (\ref{eq:alphasymm}), and the kurtosis approaches the Skellam limit as $\alpha$ nears either zero or unity. The kurtosis dips significantly compared to that of a Skellam distribution for $\alpha\sim 1/2$. 


\begin{figure}
%\centerline{\includegraphics[width=0.7\textwidth]{figs/eff_T150V100B0}}
\caption{\label{fig:eff_T150V100B0}
Blah blah...}
\end{figure}

\section{Blast Wave Model and Comparison to STAR Results}\label{sec:blast}

\section{Summary}\label{sec:summary}

\begin{acknowledgments}
This work was supported by the Department of Energy Office of Science through grant number DE-FG02-03ER41259 and through grant number DE-FG02-87ER40328. 
\end{acknowledgments}

\begin{thebibliography}{99}

%\cite{Savchuk:2019xfg}
\bibitem{Savchuk:2019xfg}
O.~Savchuk, R.~V.~Poberezhnyuk, V.~Vovchenko and M.~I.~Gorenstein,
%``Binomial acceptance corrections for particle number distributions in high-energy reactions,''
Phys. Rev. C \textbf{101} (2020) no.2, 024917
doi:10.1103/PhysRevC.101.024917
[arXiv:1911.03426 [hep-ph]].
%2 citations counted in INSPIRE as of 05 May 2020

%\cite{Oliinychenko:2020cmr}
\bibitem{Oliinychenko:2020cmr} 
  D.~Oliinychenko, S.~Shi and V.~Koch,
  %``Effects of local event-by-event conservation laws in ultra-relativistic heavy ion collisions at the particlization,''
  arXiv:2001.08176 [hep-ph].
  %%CITATION = ARXIV:2001.08176;%%
  
  %\cite{Pratt:1999ht}
\bibitem{Pratt:1999ht} 
  S.~Pratt and S.~Das Gupta,
  %``Statistical models of nuclear fragmentation,''
  Phys.\ Rev.\ C {\bf 62}, 044603 (2000)
  doi:10.1103/PhysRevC.62.044603
  [nucl-th/9903006].
  %%CITATION = doi:10.1103/PhysRevC.62.044603;%%
  %25 citations counted in INSPIRE as of 29 Mar 2020

%\cite{Cheng:2002jb}
\bibitem{Cheng:2002jb}
S.~Cheng and S.~Pratt,
%``Isospin fluctuations from a thermally equilibrated hadron gas,''
Phys.\ Rev.\ C \textbf{67}, 044904 (2003)
doi:10.1103/PhysRevC.67.044904
[arXiv:nucl-th/0207051 [nucl-th]].
%4 citations counted in INSPIRE as of 30 Mar 2020

\bibitem{Pratt:2003jd}
S.~Pratt and J.~Ruppert,
%``The Quark gluon plasma in a finite volume,''
Phys.\ Rev.\ C \textbf{68}, 024904 (2003)
doi:10.1103/PhysRevC.68.024904
[arXiv:nucl-th/0303043 [nucl-th]].
%6 citations counted in INSPIRE as of 30 Mar 2020



















%\cite{Pratt:2015zsa}
\bibitem{Pratt:2015zsa} 
  S.~Pratt, E.~Sangaline, P.~Sorensen and H.~Wang,
  %``Constraining the Eq. of State of Super-Hadronic Matter from Heavy-Ion Collisions,''
  Phys.\ Rev.\ Lett.\  {\bf 114}, 202301 (2015).
  %doi:10.1103/PhysRevLett.114.202301.
  %[arXiv:1501.04042 [nucl-th]].
  %%CITATION = doi:10.1103/PhysRevLett.114.202301;%%
  %62 citations counted in INSPIRE as of 21 Mar 2019
  
	
%\cite{Pratt:2015jsa}
\bibitem{Pratt:2015jsa} 
  S.~Pratt, W.~P.~McCormack and C.~Ratti,
  %``Production of Charge in Heavy Ion Collisions,''
  Phys.\ Rev.\ C {\bf 92}, 064905 (2015).
  %doi:10.1103/PhysRevC.92.064905
  %[arXiv:1508.07031 [nucl-th]].
  %%CITATION = doi:10.1103/PhysRevC.92.064905;%%
  %4 citations counted in INSPIRE as of 19 Dec 2017 
  
%\cite{Bernhard:2016tnd}
\bibitem{Bernhard:2016tnd} 
  J.~E.~Bernhard, J.~S.~Moreland, S.~A.~Bass, J.~Liu and U.~Heinz,
  %``Applying Bayesian parameter estimation to relativistic heavy-ion collisions: simultaneous characterization of the initial state and quark-gluon plasma medium,''
  Phys.\ Rev.\ C {\bf 94}, no. 2, 024907 (2016).
  %doi:10.1103/PhysRevC.94.024907
 % [arXiv:1605.03954 [nucl-th]].
  %%CITATION = doi:10.1103/PhysRevC.94.024907;%%
  %150 citations counted in INSPIRE as of 21 Mar 2019
  
%\cite{Bernhard:2015hxa}
\bibitem{Bernhard:2015hxa} 
  J.~E.~Bernhard, P.~W.~Marcy, C.~E.~Coleman-Smith, S.~Huzurbazar, R.~L.~Wolpert and S.~A.~Bass,
  %``Quantifying properties of hot and dense QCD matter through systematic model-to-data comparison,''
  Phys.\ Rev.\ C {\bf 91}, no. 5, 054910 (2015).
  %doi:10.1103/PhysRevC.91.054910
 % [arXiv:1502.00339 [nucl-th]].
  %%CITATION = doi:10.1103/PhysRevC.91.054910;%%
  %36 citations counted in INSPIRE as of 21 Mar 2019

%\cite{Burke:2013yra}
\bibitem{Burke:2013yra}
  K.~M.~Burke {\it et al.} [JET Collaboration],
  %``Extracting the jet transport coefficient from jet quenching in high-energy heavy-ion collisions,''
  Phys.\ Rev.\ C {\bf 90}, no. 1, 014909 (2014).
 % doi:10.1103/PhysRevC.90.014909
 % [arXiv:1312.5003 [nucl-th]].
  %%CITATION = doi:10.1103/PhysRevC.90.014909;%%
  %215 citations counted in INSPIRE as of 21 Mar 2019

%\cite{He:2018gks}
\bibitem{He:2018gks} 
  Y.~He, L.~G.~Pang and X.~N.~Wang,
  %``Bayesian extraction of jet energy loss distributions in heavy-ion collisions,''
  arXiv:1808.05310 [hep-ph].
  %%CITATION = ARXIV:1808.05310;%%
  %3 citations counted in INSPIRE as of 21 Mar 2019

%\cite{Xu:2017obm}
\bibitem{Xu:2017obm} 
  Y.~Xu, J.~E.~Bernhard, S.~A.~Bass, M.~Nahrgang and S.~Cao,
  %``Data-driven analysis for the temperature and momentum dependence of the heavy-quark diffusion coefficient in relativistic heavy-ion collisions,''
  Phys.\ Rev.\ C {\bf 97}, no. 1, 014907 (2018).
 % doi:10.1103/PhysRevC.97.014907
 % [arXiv:1710.00807 [nucl-th]].
  %%CITATION = doi:10.1103/PhysRevC.97.014907;%%
  %27 citations counted in INSPIRE as of 21 Mar 2019
  
\bibitem{Greif:2017byw} 
  M.~Greif, J.~A.~Fotakis, G.~S.~Denicol and C.~Greiner,
  Phys.\ Rev.\ Lett.\  {\bf 120}, 242301 (2018).

\bibitem{Pratt:2019fbj} 
  S.~Pratt,
  %``Calculating n-Point Charge Correlations in Evolving Systems,''
  arXiv:1908.01053 [nucl-th].

\bibitem{Borsanyi:2011sw}
  S.~Borsanyi, Z.~Fodor, S.~D.~Katz, S.~Krieg, C.~Ratti and K.~Szabo,
  %``Fluctuations of conserved charges at finite temperature from lattice QCD,''
  JHEP {\bf 1201}, 138 (2012).
 % [arXiv:1112.4416 [hep-lat]].
  %%CITATION = ARXIV:1112.4416;%%

%\cite{Aarts:2014nba}
\bibitem{Aarts:2014nba} 
  G.~Aarts, C.~Allton, A.~Amato, P.~Giudice, S.~Hands and J.~I.~Skullerud,
  %``Electrical conductivity and charge diffusion in thermal QCD from the lattice,''
  JHEP {\bf 1502}, 186 (2015).
  %doi:10.1007/JHEP02(2015)186
  %[arXiv:1412.6411 [hep-lat]].
  %%CITATION = doi:10.1007/JHEP02(2015)186;%%
  %71 citations counted in INSPIRE as of 27 Nov 2017

%\cite{Amato:2013naa}
\bibitem{Amato:2013naa} 
  A.~Amato, G.~Aarts, C.~Allton, P.~Giudice, S.~Hands and J.~I.~Skullerud,
  %``Electrical conductivity of the quark-gluon plasma across the deconfinement transition,''
  Phys.\ Rev.\ Lett.\  {\bf 111}, no. 17, 172001 (2013).
 % doi:10.1103/PhysRevLett.111.172001
 % [arXiv:1307.6763 [hep-lat]].
  %%CITATION = doi:10.1103/PhysRevLett.111.172001;%%
  %119 citations counted in INSPIRE as of 25 Mar 2019

%\cite{Policastro:2002se} diffusion from AdS-CFT
\bibitem{Policastro:2002se} 
  G.~Policastro, D.~T.~Son and A.~O.~Starinets,
  %``From AdS / CFT correspondence to hydrodynamics,''
  JHEP {\bf 0209}, 043 (2002).
 % doi:10.1088/1126-6708/2002/09/043
 % [hep-th/0205052].
  %%CITATION = doi:10.1088/1126-6708/2002/09/043;%%
  %641 citations counted in INSPIRE as of 25 Mar 2019

%\cite{CasalderreySolana:2006rq} D for heavy quarks from AdS/CFT
\bibitem{CasalderreySolana:2006rq} 
  J.~Casalderrey-Solana and D.~Teaney,
  %``Heavy quark diffusion in strongly coupled N=4 Yang-Mills,''
  Phys.\ Rev.\ D {\bf 74}, 085012 (2006).
  %doi:10.1103/PhysRevD.74.085012
  %[hep-ph/0605199].
  %%CITATION = doi:10.1103/PhysRevD.74.085012;%%
  %383 citations counted in INSPIRE as of 25 Mar 2019

%\cite{Ghiglieri:2018dib}
\bibitem{Ghiglieri:2018dib} 
  J.~Ghiglieri, G.~D.~Moore and D.~Teaney,
  %``QCD Shear Viscosity at (almost) NLO,''
  JHEP {\bf 1803}, 179 (2018).
  %doi:10.1007/JHEP03(2018)179
  %[arXiv:1802.09535 [hep-ph]].
  %%CITATION = doi:10.1007/JHEP03(2018)179;%%
  %13 citations counted in INSPIRE as of 01 May 2019
  
  
\bibitem{Greif:2016skc} 
  M.~Greif, C.~Greiner and G.~S.~Denicol,
  %``Electric conductivity of a hot hadron gas from a kinetic approach,''
  Phys.\ Rev.\ D {\bf 93}, no. 9, 096012 (2016)
  Erratum: [Phys.\ Rev.\ D {\bf 96}, no. 5, 059902 (2017)]
  
\bibitem{Hammelmann:2018ath} 
  J.~Hammelmann, J.~M.~Torres-Rincon, J.~B.~Rose, M.~Greif and H.~Elfner,
  %``Electrical conductivity and relaxation via colored noise in a hadronic gas,''
  Phys.\ Rev.\ D {\bf 99}, no. 7, 076015 (2019).

%\cite{Pratt:2018ebf}
\bibitem{Pratt:2018ebf} 
  S.~Pratt and C.~Plumberg,
  %``Evolving Charge Correlations in a Hybrid Model with both Hydrodynamics and Hadronic Boltzmann Descriptions,''
  to appear in Phys. Rev. C, arXiv:1812.05649 [nucl-th].
  %%CITATION = ARXIV:1812.05649;%%
  %1 citations counted in INSPIRE as of 11 Apr 2019

\bibitem{Shen:2014vra} 
  C.~Shen, Z.~Qiu, H.~Song, J.~Bernhard, S.~Bass and U.~Heinz,
  %``The iEBE-VISHNU code package for relativistic heavy-ion collisions,''
  Comput.\ Phys.\ Commun.\  {\bf 199}, 61 (2016)

%\cite{Pratt:2017oyf}
\bibitem{Pratt:2017oyf}
  S.~Pratt, J.~Kim and C.~Plumberg,
  %``Evolution of Charge Fluctuations and Correlations in the Hydrodynamic Stage of Heavy Ion Collisions,''
  Phys.\ Rev.\ C {\bf 98}, no. 1, 014904 (2018).
  %doi:10.1103/PhysRevC.98.014904.
%  [arXiv:1712.09298 [nucl-th]].
  %%CITATION = doi:10.1103/PhysRevC.98.014904;%%

%\cite{Wang:2012jua}
\bibitem{Wang:2012jua} %% Hui thesis
  H.~Wang, Ph.D. Thesis,
  %``Study of particle ratio fluctuations and charge balance functions at RHIC,''
  arXiv:1304.2073 [nucl-ex].
  %%CITATION = ARXIV:1304.2073;%%
  %4 citations counted in INSPIRE as of 29 Oct 2017

%\cite{Abelev:2010ab}
\bibitem{Abelev:2010ab} 
  B.~I.~Abelev {\it et al.} [STAR Collaboration],
  %``Longitudinal scaling property of the charge balance function in Au + Au collisions at 200 GeV,''
  Phys.\ Lett.\ B {\bf 690}, 239 (2010).
  %doi:10.1016/j.physletb.2010.05.028
  %[arXiv:1002.1641 [nucl-ex]].
  %%CITATION = doi:10.1016/j.physletb.2010.05.028;%%
  %10 citations counted in INSPIRE as of 14 Dec 2017

%\cite{Li:2011zzx}
%\bibitem{Li:2011zzx} 
  N.~Li {\it et al.} [STAR Collaboration],
  %``The study of longitudinal properties of the charge balance function,''
%  Indian J.\ Phys.\  {\bf 85}, 923 (2011).
  %doi:10.1007/s12648-011-0100-0
  %%CITATION = doi:10.1007/s12648-011-0100-0;%%

%\cite{Adams:2003kg}
\bibitem{Adams:2003kg} 
  J.~Adams {\it et al.} [STAR Collaboration],
  %``Narrowing of the balance function with centrality in au + au collisions at (S(NN))**1/2 = 130-GeV,''
  Phys.\ Rev.\ Lett.\  {\bf 90}, 172301 (2003).
  %doi:10.1103/PhysRevLett.90.172301
  %[nucl-ex/0301014].
  %%CITATION = doi:10.1103/PhysRevLett.90.172301;%%
  %88 citations counted in INSPIRE as of 14 Dec 2017 
   
%\cite{Aggarwal:2010ya}
\bibitem{Aggarwal:2010ya} 
  M.~M.~Aggarwal {\it et al.} [STAR Collaboration],
  %``Balance Functions from Au$+$Au, $d+$Au, and $p+p$ Collisions at $\sqrt{s_{NN}}$ = 200 GeV,''
  Phys.\ Rev.\ C {\bf 82}, 024905 (2010).
  %doi:10.1103/PhysRevC.82.024905
  %[arXiv:1005.2307 [nucl-ex]].
  %%CITATION = doi:10.1103/PhysRevC.82.024905;%%
  %43 citations counted in INSPIRE as of 14 Dec 2017

%\cite{Alt:2004gx}
%\bibitem{Alt:2004gx} 
 % C.~Alt {\it et al.} [NA49 Collaboration],
  %``System size and centrality dependence of the balance function in A + A collisions at s(NN)**(1/2) = 17.2-GeV,''
 % Phys.\ Rev.\ C {\bf 71}, 034903 (2005).
  %doi:10.1103/PhysRevC.71.034903
  %[hep-ex/0409031].
  %%CITATION = doi:10.1103/PhysRevC.71.034903;%%
  %36 citations counted in INSPIRE as of 14 Dec 2017

%\cite{Abelev:2013csa}
\bibitem{Abelev:2013csa}
  B.~Abelev {\it et al.} [ALICE Collaboration],
  %``Charge correlations using the balance function in Pb-Pb collisions at $\sqrt{s_{NN}}$ = 2.76 TeV,''
  Phys.\ Lett.\ B {\bf 723}, 267 (2013).
  %doi:10.1016/j.physletb.2013.05.039
  %[arXiv:1301.3756 [nucl-ex]].
  %%CITATION = doi:10.1016/j.physletb.2013.05.039;%%
  %22 citations counted in INSPIRE as of 14 Dec 2017
  %measured both for Delta phi and Delta eta
  
%\cite{Alt:2007hk}
\bibitem{Alt:2007hk} 
  C.~Alt {\it et al.} [NA49 Collaboration],
  %``Rapidity and energy dependence of the electric charge correlations in A + A collisions at the SPS energies,''
  Phys.\ Rev.\ C {\bf 76}, 024914 (2007).
  %doi:10.1103/PhysRevC.76.024914
  %[arXiv:0705.1122 [nucl-ex]].
  %%CITATION = doi:10.1103/PhysRevC.76.024914;%%
  %19 citations counted in INSPIRE as of 14 Dec 2017

%\cite{Adamczyk:2015yga}
\bibitem{Adamczyk:2015yga} 
  L.~Adamczyk {\it et al.} [STAR Collaboration],
  %``Beam-energy dependence of charge balance functions from Au + Au collisions at energies available at the BNL Relativistic Heavy Ion Collider,''
  Phys.\ Rev.\ C {\bf 94}, no. 2, 024909 (2016).
  %doi:10.1103/PhysRevC.94.024909
  %[arXiv:1507.03539 [nucl-ex]].
  %%CITATION = doi:10.1103/PhysRevC.94.024909;%%
  %3 citations counted in INSPIRE as of 29 Oct 2017

%\cite{Adamczyk:2013hsi}
\bibitem{Adamczyk:2013hsi} 
  L.~Adamczyk {\it et al.} [STAR Collaboration],
  %``Fluctuations of charge separation  perpendicular to the event plane and local parity violation in $\sqrt{s_{NN}}=200$ GeV Au+Au  collisions at the BNL Relativistic Heavy Ion Collider,''
  Phys.\ Rev.\ C {\bf 88}, no. 6, 064911 (2013).
 % doi:10.1103/PhysRevC.88.064911.
%  [arXiv:1302.3802 [nucl-ex]].
  %%CITATION = doi:10.1103/PhysRevC.88.064911;%%
  %57 citations counted in INSPIRE as of 20 Nov 2018

%\cite{Abelev:2009ac}
\bibitem{Abelev:2009ac} 
  B.~I.~Abelev {\it et al.} [STAR Collaboration],
  %``Azimuthal Charged-Particle Correlations and Possible Local Strong Parity Violation,''
  Phys.\ Rev.\ Lett.\  {\bf 103}, 251601 (2009).
 % doi:10.1103/PhysRevLett.103.251601.
%  [arXiv:0909.1739 [nucl-ex]].
  %%CITATION = doi:10.1103/PhysRevLett.103.251601;%%
  %397 citations counted in INSPIRE as of 20 Nov 2018

\bibitem{Bass:2000az} 
  S.~A.~Bass, P.~Danielewicz and S.~Pratt,
  %``Clocking hadronization in relativistic heavy ion collisions with balance functions,''
  Phys.\ Rev.\ Lett.\  {\bf 85}, 2689 (2000).

%\cite{Pratt:2012dz}
\bibitem{Pratt:2012dz} 
  S.~Pratt,
  %``Identifying the Charge Carriers of the Quark-Gluon Plasma,''
  Phys.\ Rev.\ Lett.\  {\bf 108}, 212301 (2012).
  %doi:10.1103/PhysRevLett.108.212301
  %[arXiv:1203.4578 [nucl-th]].
  %%CITATION = doi:10.1103/PhysRevLett.108.212301;%%
  %22 citations counted in INSPIRE as of 19 Dec 2017


%\cite{Pan:2014caa}
\bibitem{Pan:2014caa} 
  Y.~Pan and S.~Pratt,
  %``Baryon annihilation and regeneration in heavy ion collisions,''
  Phys.\ Rev.\ C {\bf 89}, no. 4, 044911 (2014).
  %doi:10.1103/PhysRevC.89.044911
  %%CITATION = doi:10.1103/PhysRevC.89.044911;%%
  %10 citations counted in INSPIRE as of 19 Dec 2017

%\cite{Steinheimer:2017vju}
\bibitem{Steinheimer:2017vju} 
  J.~Steinheimer, J.~Aichelin, M.~Bleicher and H.~Stöcker,
  %``Influence of the hadronic phase on observables in ultrarelativistic heavy ion collisions,''
  Phys.\ Rev.\ C {\bf 95}, no. 6, 064902 (2017).
  %doi:10.1103/PhysRevC.95.064902
  %[arXiv:1703.06638 [nucl-th]].
  %%CITATION = doi:10.1103/PhysRevC.95.064902;%%
  %5 citations counted in INSPIRE as of 19 Dec 2017
 
%\cite{Steinheimer:2012rd}
\bibitem{Steinheimer:2012rd} 
  J.~Steinheimer, J.~Aichelin and M.~Bleicher,
  %``Nonthermal p/? Ratio at LHC as a Consequence of Hadronic Final State Interactions,''
  Phys.\ Rev.\ Lett.\  {\bf 110}, no. 4, 042501 (2013).
  %doi:10.1103/PhysRevLett.110.042501
  %[arXiv:1203.5302 [nucl-th]].
  %%CITATION = doi:10.1103/PhysRevLett.110.042501;%%
  %84 citations counted in INSPIRE as of 19 Dec 2017
  
%%%%%%%%%%%%%%%%%%%%%%% XXXXXXX

\begin{comment}

%\cite{Bellwied:2015lba}
%\bibitem{Bellwied:2015lba} 
  R.~Bellwied, S.~Borsanyi, Z.~Fodor, S.~D.~Katz, A.~Pasztor, C.~Ratti and K.~K.~Szabo,
  %``Fluctuations and correlations in high temperature QCD,''
  Phys.\ Rev.\ D {\bf 92}, no. 11, 114505 (2015).
 % doi:10.1103/PhysRevD.92.114505.
%  [arXiv:1507.04627 [hep-lat]].
  %%CITATION = doi:10.1103/PhysRevD.92.114505;%%
  %74 citations counted in INSPIRE as of 28 Mar 2018


%\bibitem{Ling:2013ksb} 
  B.~Ling, T.~Springer and M.~Stephanov,
  %``Hydrodynamics of charge fluctuations and balance functions,''
  Phys.\ Rev.\ C {\bf 89}, no. 6, 064901 (2014).

%\cite{Bozek:2004dt}
%\bibitem{Bozek:2004dt} 
  P.~Bozek,
  %``The Balance functions in azimuthal angle is a measure of the transverse flow,''
  Phys.\ Lett.\ B {\bf 609}, 247 (2005).
  %doi:10.1016/j.physletb.2005.01.072
  %[nucl-th/0412076].
  %%CITATION = doi:10.1016/j.physletb.2005.01.072;%%
  %29 citations counted in INSPIRE as of 19 Dec 2017


%\cite{Cheng:2004zy}
%\bibitem{Cheng:2004zy} 
  S.~Cheng, S.~Petriconi, S.~Pratt, M.~Skoby, C.~Gale, S.~Jeon, V.~Topor Pop and Q.~H.~Zhang,
  %``Statistical and dynamic models of charge balance functions,''
  Phys.\ Rev.\ C {\bf 69}, 054906 (2004).
  %doi:10.1103/PhysRevC.69.054906
  %[nucl-th/0401008].
  %%CITATION = doi:10.1103/PhysRevC.69.054906;%%
  %35 citations counted in INSPIRE as of 19 Dec 2017

%\bibitem{popcorn}
	P. Ed\'en and G. Gustafson, Zeit. f\"ur Phys. C, {\bf 75}, 41 (1997).

%\bibitem{urqmd}
	S.A.Bass et al [URQMD], Progr. Part. Nucl. Physics Vol. \textbf{41}, 225 (1998).

%\cite{Alt:2007hk}
%\bibitem{Alt:2007hk} 
  C.~Alt {\it et al.} [NA49 Collaboration],
  %``Rapidity and energy dependence of the electric charge correlations in A + A collisions at the SPS energies,''
  Phys.\ Rev.\ C {\bf 76}, 024914 (2007).
  %doi:10.1103/PhysRevC.76.024914
  %[arXiv:0705.1122 [nucl-ex]].
  %%CITATION = doi:10.1103/PhysRevC.76.024914;%%
  %19 citations counted in INSPIRE as of 14 Dec 2017

%\cite{Adamczyk:2015yga}
%\bibitem{Adamczyk:2015yga} 
  L.~Adamczyk {\it et al.} [STAR Collaboration],
  %``Beam-energy dependence of charge balance functions from Au + Au collisions at energies available at the BNL Relativistic Heavy Ion Collider,''
  Phys.\ Rev.\ C {\bf 94}, no. 2, 024909 (2016).
  %doi:10.1103/PhysRevC.94.024909
  %[arXiv:1507.03539 [nucl-ex]].
  %%CITATION = doi:10.1103/PhysRevC.94.024909;%%
  %3 citations counted in INSPIRE as of 29 Oct 2017

%\cite{Pan:2015pzh}
%\bibitem{Pan:2015pzh} 
  Y.~H.~Pan and W.~N.~Zhang,
  %``Charge balance functions in a scenario of continuing charge production in quark matter,''
  Eur.\ Phys.\ J.\ A {\bf 51}, no. 11, 147 (2015).
  %doi:10.1140/epja/i2015-15147-3
  %%CITATION = doi:10.1140/epja/i2015-15147-3;%%

%\bibitem{Heinz:2013wva} 
  U.~W.~Heinz,
  %``Towards the Little Bang Standard Model,''
  J.\ Phys.\ Conf.\ Ser.\  {\bf 455}, 012044 (2013).

%\bibitem{sorgepionwind}
	H. Sorge, Phys. Lett. B 373, 16 􏰞1993􏰀.
	
%\cite{Pratt:1998gt}
%\bibitem{Pratt:1998gt} 
  S.~Pratt and J.~Murray,
  %``Modeling the breakup stage of relativistic heavy ion collisions,''
  Phys.\ Rev.\ C {\bf 57}, 1907 (1998).
  %doi:10.1103/PhysRevC.57.1907
  %%CITATION = doi:10.1103/PhysRevC.57.1907;%%
  %23 citations counted in INSPIRE as of 27 Nov 2018
 

%\cite{Novak:2013bqa}
%\bibitem{Novak:2013bqa} 
  J.~Novak, K.~Novak, S.~Pratt, J.~Vredevoogd, C.~Coleman-Smith and R.~Wolpert,
  %``Determining Fundamental Properties of Matter Created in Ultrarelativistic Heavy-Ion Collisions,''
  Phys.\ Rev.\ C {\bf 89}, no. 3, 034917 (2014).
 % doi:10.1103/PhysRevC.89.034917.
 % [arXiv:1303.5769 [nucl-th]].
  %%CITATION = doi:10.1103/PhysRevC.89.034917;%%
  %61 citations counted in INSPIRE as of 29 Nov 2018 




%\cite{Shen:2014vra}
%\bibitem{Shen:2014vra} 
  C.~Shen, Z.~Qiu, H.~Song, J.~Bernhard, S.~Bass and U.~Heinz,
  %``The iEBE-VISHNU code package for relativistic heavy-ion collisions,''
  Comput.\ Phys.\ Commun.\  {\bf 199}, 61 (2016).

%\bibitem{Huovinen:2009yb} 
  P.~Huovinen and P.~Petreczky,
  %``QCD Equation of State and Hadron Resonance Gas,''
  Nucl.\ Phys.\ A {\bf 837}, 26 (2010).

%\bibitem{Kapusta:2014dja} 
  J.~I.~Kapusta and C.~Young,
  %``Causal Baryon Diffusion and Colored Noise,''
  Phys.\ Rev.\ C {\bf 90}, no. 4, 044902 (2014).

%\cite{Kapusta:2017hfi}
%\bibitem{Kapusta:2017hfi} 
  J.~I.~Kapusta and C.~Plumberg,
  %``Causal Electric Charge Diffusion and Balance Functions in Relativistic Heavy Ion Collisions,''
  Phys.\ Rev.\ C {\bf 97}, no. 1, 014906 (2018).
 % doi:10.1103/PhysRevC.97.014906.
 % [arXiv:1710.03329 [nucl-th]].
  %%CITATION = doi:10.1103/PhysRevC.97.014906;%%
  %8 citations counted in INSPIRE as of 13 Dec 2018
  
%\cite{Aziz:2004qu}
%\bibitem{Aziz:2004qu} 
  M.~A.~Aziz and S.~Gavin,
  %``Causal diffusion and the survival of charge fluctuations in nuclear collisions,''
  Phys.\ Rev.\ C {\bf 70}, 034905 (2004).
%  doi:10.1103/PhysRevC.70.034905.
 % [nucl-th/0404058].
  %%CITATION = doi:10.1103/PhysRevC.70.034905;%%
  %58 citations counted in INSPIRE as of 28 Nov 2018

%\bibitem{Cooper:1974mv} 
  F.~Cooper and G.~Frye,
  %``Comment on the Single Particle Distribution in the Hydrodynamic and Statistical Thermodynamic Models of Multiparticle Production,''
  Phys.\ Rev.\ D {\bf 10}, 186 (1974).

%\bibitem{Huovinen:2012is} 
  P.~Huovinen and H.~Petersen,
  %``Particlization in hybrid models,''
  Eur.\ Phys.\ J.\ A {\bf 48}, 171 (2012).

%\cite{Becattini:2012sq}
%\bibitem{Becattini:2012sq} 
  F.~Becattini, M.~Bleicher, T.~Kollegger, M.~Mitrovski, T.~Schuster and R.~Stock,
  %``Hadronization and Hadronic Freeze-Out in Relativistic Nuclear Collisions,''
  Phys.\ Rev.\ C {\bf 85}, 044921 (2012).
  %doi:10.1103/PhysRevC.85.044921
  %[arXiv:1201.6349 [nucl-th]].
  %%CITATION = doi:10.1103/PhysRevC.85.044921;%%
  %59 citations counted in INSPIRE as of 19 Dec 2017

%\cite{Becattini:2012xb}
%\bibitem{Becattini:2012xb} 
  F.~Becattini, M.~Bleicher, T.~Kollegger, T.~Schuster, J.~Steinheimer and R.~Stock,
  %``Hadron Formation in Relativistic Nuclear Collisions and the QCD Phase Diagram,''
  Phys.\ Rev.\ Lett.\  {\bf 111}, 082302 (2013).
  %doi:10.1103/PhysRevLett.111.082302
  %[arXiv:1212.2431 [nucl-th]].
  %%CITATION = doi:10.1103/PhysRevLett.111.082302;%%
  %110 citations counted in INSPIRE as of 19 Dec 2017

%\cite{Pratt:2014vja}
%\bibitem{Pratt:2014vja} 
  S.~Pratt,
  %``Accounting for backflow in hydrodynamic-Boltzmann interfaces,''
  Phys.\ Rev.\ C {\bf 89}, no. 2, 024910 (2014).
  %doi:10.1103/PhysRevC.89.024910
  %[arXiv:1401.0316 [nucl-th]]
  %%CITATION = doi:10.1103/PhysRevC.89.024910;%%
  %8 citations counted in INSPIRE as of 19 Dec 2017
%
%\bibitem{Pratt:2010jt} 
  S.~Pratt and G.~Torrieri,
  %``Coupling Relativistic Viscous Hydrodynamics to Boltzmann Descriptions,''
  Phys.\ Rev.\ C {\bf 82}, 044901 (2010).

%\cite{Bozek:2003qi}
%\bibitem{Bozek:2003qi} 
  P.~Bozek, W.~Broniowski and W.~Florkowski,
  %``Balance functions in a thermal model with resonances,''
  Acta Phys.\ Hung.\ A {\bf 22}, 149 (2005).
  %doi:10.1556/APH.22.2005.1-2.15
  %[nucl-th/0310062].
  %%CITATION = doi:10.1556/APH.22.2005.1-2.15;%%
  %35 citations counted in INSPIRE as of 19 Dec 2017

%\bibitem{WestfallAcceptance}
% Routines for modeling the efficiency of the STAR detector were generously provided by Gary Westfall.

%\cite{Steinheimer:2012bn}
%\bibitem{Steinheimer:2012bn} 
  J.~Steinheimer, V.~Koch and M.~Bleicher,
  %``Hydrodynamics at large baryon densities: Understanding proton vs. anti-proton v_2 and other puzzles,''
  Phys.\ Rev.\ C {\bf 86}, 044903 (2012).
  %doi:10.1103/PhysRevC.86.044903
  %[arXiv:1207.2791 [nucl-th]].
  %%CITATION = doi:10.1103/PhysRevC.86.044903;%%
  %36 citations counted in INSPIRE as of 19 Dec 2017

%\cite{Pratt:2003gh}
%\bibitem{Pratt:2003gh} 
  S.~Pratt and S.~Cheng,
  %``Removing distortions from charge balance functions,''
  Phys.\ Rev.\ C {\bf 68}, 014907 (2003).
  %doi:10.1103/PhysRevC.68.014907.
  %[nucl-th/0303025].
  %%CITATION = doi:10.1103/PhysRevC.68.014907;%%
  %28 citations counted in INSPIRE as of 29 Nov 2018
 
%\cite{Schlichting:2010qia}
%\bibitem{Schlichting:2010qia} 
  S.~Schlichting and S.~Pratt,
  %``Charge conservation at energies available at the BNL Relativistic Heavy Ion Collider and contributions to local parity violation observables,''
  Phys.\ Rev.\ C {\bf 83}, 014913 (2011).
 % doi:10.1103/PhysRevC.83.014913.
 % [arXiv:1009.4283 [nucl-th]].
  %%CITATION = doi:10.1103/PhysRevC.83.014913;%%
  %120 citations counted in INSPIRE as of 30 Nov 2018

%\cite{Pratt:2010zn}
%\bibitem{Pratt:2010zn} 
  S.~Pratt, S.~Schlichting and S.~Gavin,
  %``Effects of Momentum Conservation and Flow on Angular Correlations at RHIC,''
  Phys.\ Rev.\ C {\bf 84}, 024909 (2011).
  %doi:10.1103/PhysRevC.84.024909.
  %[arXiv:1011.6053 [nucl-th]].
  %%CITATION = doi:10.1103/PhysRevC.84.024909;%%
  %74 citations counted in INSPIRE as of 30 Nov 2018  

%\bibitem{Gelis:2013rba} 
  T.~Epelbaum and F.~Gelis,
  %``Pressure isotropization in high energy heavy ion collisions,''
  Phys.\ Rev.\ Lett.\  {\bf 111}, 232301 (2013).

%\bibitem{Dusling:2010rm} 
  K.~Dusling, T.~Epelbaum, F.~Gelis and R.~Venugopalan,
  %``Role of quantum fluctuations in a system with strong fields: Onset of hydrodynamical flow,''
  Nucl.\ Phys.\ A {\bf 850}, 69 (2011).

%\bibitem{Vredevoogd:2008id} 
  J.~Vredevoogd and S.~Pratt,
  %``Universal Flow in the First Stage of Relativistic Heavy Ion Collisions,''
  Phys.\ Rev.\ C {\bf 79}, 044915 (2009).
 % doi:10.1103/PhysRevC.79.044915.
 % [arXiv:0810.4325 [nucl-th]].
  %%CITATION = doi:10.1103/PhysRevC.79.044915;%%
  %80 citations counted in INSPIRE as of 02 Dec 2018

\end{comment}
         
\end{thebibliography}

\end{document}
