\documentclass[aps,prc,nofootinbib,showpacs,superscriptaddress,groupedaddress]{revtex4-1}
\usepackage{amsmath,amssymb,amsbsy,bm}
\usepackage{graphicx}
\usepackage{comment}
\usepackage{float}
%\usepackage{mathabx}
\usepackage[colorlinks=true,linkcolor=blue,citecolor=blue,urlcolor=blue]{hyperref}

\begin{document}

\title{Charge Conservation and Higher Moments of Charge Fluctuations}
\author{Scott Pratt}
\affiliation{Department of Physics and Astronomy and National Superconducting Cyclotron Laboratory\\
Michigan State University, East Lansing, MI 48824~~USA}
\author{Rachel Steinhorst}
\affiliation{Department of Physics and Astronomy\\
Michigan State University, East Lansing, MI 48824~~USA}
\date{\today}

\pacs{}

\begin{abstract}
Higher moments of distributions of net charge and baryon number in heavy-ion collisions have been proposed as signals of fundamental QCD phase transitions. In order to better understand background processes for these observables, models are presented which enable one to gauge the effects of local charge conservation, Bose symmetrization, and volume fluctuations. The models consist of a thermal model superimposed onto a simple parameterization of collective flow, known as a blast-wave, with emission being consistent with individual canonical ensembles. The spatial extent of local charge conservation is parameterized by a patch volume. The sensitivity of third and fourth order moments, skewness and kurtosis, to the patch volume and other parameters, including beam energy and average net baryon multiplicity is explored. Comparisons with STAR data show that a significant part of the observed non-Possonian fluctuations in net-proton fluctuations are explained by charge-conservation, but that measurements of the STAR collaboration for fluctuations of net electric charge qualitatively differ from expectations of the models presented here.
\end{abstract}

\maketitle

\input intro.tex

\input cumulantdefs.tex

\input theory.tex

\input mctheory.tex

\input bosefermi.tex

\input uniformalpha.tex

\input blastwave.tex

\input summary.tex

\begin{acknowledgments}
This work was supported by the Department of Energy Office of Science through grant number DE-FG02-03ER41259 and through grant number DE-FG02-87ER40328. R. Steinhorst was additionally supported by the MSU Professorial Assistantship program and by the Director's Research Scholars program at the National Superconducting Cyclotron Laboratory. The work benefited by discussions within the BEST Collaboration, which is a topical collaboration funded by the Department of Energy Office of Science, and especially from discussions with Dmytro Oliinychenko.
\end{acknowledgments}

\input biblio.tex

\end{document}
