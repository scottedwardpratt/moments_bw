% !TEX root =  CCmoments.tex

\section{Blast Wave Model with a Full Hadron Gas and Comparison to STAR Results}\label{sec:blast}

The calculations of the previous section were based on a simple picture, where each sub-volume emitted particles whose probability of being observed was uniform, denoted by $\alpha$. In practice, this probability depends on where the sub-volume is located within the overall reaction volume. A sub-volume in the region with spatial rapidity, $|\eta_s|>1$, emits particles that are only rarely detected in by detectors that specialize in mid-rapidity measurements. Even for a sub-volume centered at mid-rapidity, thermal motion can spread some of its emission to rapidities outside the acceptance. This is especially true when the experiments narrow their acceptance. For example, STAR's acceptance nominally covers $\pm 1$ units of pseudo-rapidity, but the acceptance for identified particles is confined to $\pm 0.9$ units. For real rapidity, the effective acceptance is narrowed further, due to the fact that the true rapidity $y$ is less than the pseudo-rapidity $\eta$. This difference magnifies for massive, or slower particles. In fact, STAR analyses of identified particles often make cuts and only consider particles with true rapidities $-0.5<y<0.5$. Of course, even for particles within the rapidity acceptance, the particles must exceed some minimum transverse momentum and the efficiency for being detected is imperfect. For particles identified only by charge, the efficiencies are typically $\gtrsim 80\%$, and for identified particles the efficiency falls by another few tens of percent.

Collective flow plays a critical role. First, longitudinal flow is what allows the measurement of rapidity to correlate to a measurement in coordinate-space rapidity. In the Bjorken model of boost invariant flow, the mapping is simple in that a fluid element with spatial rapidity $\eta_s$ moves with a rapidity $\eta_s$, and particles emitted from a sub-volume with rapidity $\eta_s$ have rapidities $y\approx\eta_s$. The mapping is smeared by the thermal motion. Collective radial flow and cooling combine to better align $\eta_s$ and $y$. The thermal spread for pions is $\approx 0.6$ units of rapidity, while that for protons is $\approx 0.25$.

The size of the sub-volume clearly drives the result. If the sub-volume is small, there is an enhanced probability that a charge and its balancing charge will both be identified and cancel one another. If the sub-volume extends over a large rapidity range, the effects of charge balance are minimized. The extent of the sub-volume in the transverse direction is less important. For sub-volumes at the edge of the fireball, which have more collective velocity, there is a modest increase of having balancing charges both pushed into the acceptance, and avoid the low $p_t$ cutoff, but that sensitivity is modest compared to the extent of the sub-volume in spatial rapidity. The transverse size also plays a role, as illustrated in the previous session, in that the canonical suppression is important for smaller overall volumes. 

The intent of this study is to overlay canonical sub-volumes onto a blast-wave model of collective flow and to filter the calculations with the STAR acceptance and efficiency. This should be sufficiently realistic to make meaningful comparisons to STAR data. The three main parameters for the blast-wave model typically describe the radial flow $u_\perp$, the kinetic freeze-out temperature, $T_k$, and the temperature of chemical freeze-out, $T_c$. The chemical freeze-out temperature, $T_c$, is chosen to fit relative particle yields, while $T_k$ and $u_\perp$ are determined by simultaneously fitting the spectra of species with varying mass, typically $\pi,K,p$. A variety of parameterizations exist, such as having the velocity increase linearly from the origin, or the transverse rapidity, or having a sharp cutoff in radius vs. assuming a Gaussian profile. Depending on the choice, the value of $T_k$ and $u_\perp$ vary, but are typically in the neighborhood of $T_k\approx 100$ MeV and $u_\perp\approx 0.6$. For increasing multiplicities, the reaction volumes can expand and cool further, which leads to modestly increased values of $u_\perp$ and modestly decreased values of $T_k$ for either more central or more energetic collisions.