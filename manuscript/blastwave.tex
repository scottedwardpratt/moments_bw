% !TEX root =  CCmoments.tex

\section{Blast Wave Model with a Full Hadron Gas and Comparison to STAR Results}\label{sec:blast}

The calculations of the previous section were based on a simple picture, where each sub-volume emitted particles whose probability of being observed was uniform, denoted by $\alpha$. In practice, this probability depends on where the sub-volume is located within the overall reaction volume. A sub-volume in the region with spatial rapidity, $|\eta_s|>1$, emits particles that are only rarely detected in by detectors that specialize in mid-rapidity measurements. Even for a sub-volume centered at mid-rapidity, thermal motion can spread some of its emission to rapidities outside the acceptance. This is especially true when the experiments narrow their acceptance. For example, STAR's acceptance nominally covers $\pm 1$ units of pseudo-rapidity, but the acceptance for identified particles is confined to $\pm 0.9$ units. For real rapidity, the effective acceptance is narrowed further, due to the fact that the true rapidity $y$ is less than the pseudo-rapidity $\eta$. This difference magnifies for massive, or slower particles. In fact, STAR analyses of identified particles often make cuts and only consider particles with true rapidities $-0.5<y<0.5$. Of course, even for particles within the rapidity acceptance, the particles must exceed some minimum transverse momentum and the efficiency for being detected is imperfect. For particles identified only by charge, the efficiencies are typically $\gtrsim 80\%$, and for identified particles the efficiency falls by another few tens of percent.

Collective flow plays a critical role. First, longitudinal flow is what allows the measurement of rapidity to correlate to a measurement in coordinate-space rapidity. In the Bjorken model of boost invariant flow, the mapping is simple in that a fluid element with spatial rapidity $\eta_s$ moves with a rapidity $\eta_s$, and particles emitted from a sub-volume with rapidity $\eta_s$ have rapidities $y\approx\eta_s$. The mapping is smeared by the thermal motion. Collective radial flow and cooling combine to better align $\eta_s$ and $y$. The thermal spread for pions is $\approx 0.6$ units of rapidity, while that for protons is $\approx 0.25$.

The size of the sub-volume clearly drives the result. If the sub-volume is small, there is an enhanced probability that a charge and its balancing charge will both be identified and cancel one another. If the sub-volume extends over a large rapidity range, the effects of charge balance are minimized. The extent of the sub-volume in the transverse direction is less important. For sub-volumes at the edge of the fireball, which have more collective velocity, there is a modest increase of having balancing charges both pushed into the acceptance, and avoid the low $p_t$ cutoff, but that sensitivity is modest compared to the extent of the sub-volume in spatial rapidity. The transverse size also plays a role, as illustrated in the previous session, in that the canonical suppression is important for smaller overall volumes. 

The intent of this study is to overlay canonical sub-volumes onto a blast-wave model of collective flow and to filter the calculations with the STAR acceptance and efficiency. This should be sufficiently realistic to make meaningful comparisons to STAR data. The three main parameters for the blast-wave model typically describe the radial flow $u_\perp$, the kinetic freeze-out temperature, $T_k$, and the temperature of chemical freeze-out, $T_c$. The chemical freeze-out temperature, $T_c$, is chosen to fit relative particle yields, while $T_k$ and $U_\perp$ are determined by simultaneously fitting the spectra of species with varying mass, typically $\pi,K,p$. A variety of parameterizations exist, such as having the velocity increase linearly from the origin, or the transverse rapidity, or having a sharp cutoff in radius vs. assuming a Gaussian profile. Depending on the choice, the value of $T_k$ and $u_\perp$ vary, but are typically in the neighborhood of $T_k\approx 100$ MeV and $u_\perp\approx 0.6$. For increasing multiplicities, the reaction volumes can expand and cool further, which leads to modestly increased values of $u_\perp$ and modestly decreased values of $T_k$ for either more central or for more energetic collisions. For this study the chemical freeze-out temperatures were taken from \cite{}, which extracted $T_c$ for a variety of beam energies. For this section, Bose statistics were ignored. Over 300 hadron species were included in the analysis. The flow and collective velocities were .....

Two additional parameters were added to the blast wave model. The first is the size of the sub-volume, $\Omega_c$, which was set to either 100, 200 or 500 fm$^3$, with 200 fm$^3$ being the default. This volume was used to generate the particles at $T_c$ according to the Monte Carlo procedure outline in Sec. \ref{{sec:theoryMC}, which generates independent samples consistent with the canonical ensemble. The second additional parameter describes the size of a sub-volume in spatial rapidity at the time of final emission. Those particles created from a given sub-volume were assumed to spread out and be emitted from some region with a Gaussian spread in spatial rapidity, 
\begin{eqnarray}
S(\eta_s)&\propto&\exp\left\{-\frac{(\eta_s-\eta_0)^2}{2\sigma_\eta^2}\right\}.
\end{eqnarray}
Here $\eta_0$ is a uniformly distributed random value, with the uniformity being motivated by the assumption of boost invariance. The parameter $\sigma_\eta$ describes how far particles created from the same sub-volume may have separated by the time of emission. For increasingly larger values of $\sigma_eta$, the chance that any two observed particles are correlated decreases.  One goal of this section is to evaluate the sensitivity of fluctuations to $\Omega_c$ and $\sigma_\eta$. For this specific blast-wave model, each sub-volume was assigned a radial transverse velocity according to a Gaussian,
\begin{eqnarray}
\frac{dN}{d^2u_\perp}&\approx&e^{-(u_x^2+u_y^2)/2u_\perp^2}.
\end{eqnarray}

The calculations of this section are filtered through a simplified model of the STAR detector's acceptance and efficiency. For unidentified particles, pseudo-rapidities are required to be less than 1.0 and transverse momenta are constrained to being above 150 MeV/$c$. An efficiency of 0.8 is applied. For identified particles, additional cuts are added. Rapidities are required to be less than 0.9 and transverse momenta are cut off above 1.5 GeV/$c$. The efficiency is lowered to 0.6. In order to compare results to measurement from STAR the baryon densities and chemical freeze-out temperatures were mapped to beam energy according to the analysis of \cite{....}, which extracted chemical freeze-out temperatures and chemical potentials by considering ratios of particle yields. The parameter, $u_\perp$, which controls the transverse radial flow was chosen along with the kinetic breakup temperature, $T_k$, to simultaneously fit the mean transverse momenta of both the protons and pions. Decays were simulated, and the products of weak decays (aside from pions or charged kaons) were included in the analysis. Undoubtedly, a more realistic model of the acceptance might change the ratios, but given that these are ratios, and that the overall efficiency and acceptances are not wildly off, a more rigorous model of the acceptance is unlikely to change the result by more than a few percent.

Figure \ref{fig:bw_vsvolume} illustrates how results are sensitive to the size of the sub-volume. For a volume of 100 fm$^3$ the average number of hadrons is several dozen. For very small sub-volumes, where in a grand canonical ensemble the typical number of charges would be zero or one, the thermodynamic cost of having a second charge to balance the first charge reduces the probability of having any charges, and thus lowers both the multiplicities and the moments. For larger sub-volumes, the thermodynamic cost vanishes as the system might have had fewer balancing charges of the same sign, rather than an extra balancing charge of the opposite sign. The characteristic volume for the disappearance of canonical suppression is the volume where the mean number of pairs exceeds unit. For charged particle this sets in at around 10 fm$^3$, but for baryons the characteristic volume is closer to 100 $fm^3$ because of their being heavier and fewer. Thus, the baryon moments for volumes of 100 fm$^3$ and 200 $fm^3$ differ more noticeably. The time for a fluid element to expand and cool to thermal equilibrium tends to be on the order of 5 fm/$c$, less time on the periphery, at a lower beam energy or at less centrality, and more time for a fluid element at the center, at higher beam energy, or in a more central collision. The maximum transverse distance a charge can travel is 5 fm, but if charge moves diffusively, the separation should be significantly less. As discussed below, charge can separate further along the beam axis, and because that is not well understood, the size of the sub-volume carries a large uncertainty. Anywhere from 50 fm$^3$ to a few hundred fm$^3$ might be reasonable. 

Charge can spread further in the longitudinal direction, but that distance depends sensitively on when the charges are created. Matter thermalizes at a point with large collective velocity gradients along the beam axis. If the motion is diffusive, the separation along the beam axis depends logarithmically on the ratio of the final time to the initial creation time.  Thus, if a pair are created at 0.2 fm/$c$, there separation for times, $0.2<\tau<1.0$ fm/$c$ are as important as the additional separation they gain from $1.0<\tau<5.0$ fm/$c$. For this reason the size of charge spread, $\sigma_\eta$, in spatial rapidity might be anywhere between a few tenths of a unit of rapidity to a full unit. Figure \ref{fig:bw_vssigmaeta} shows the sensitivity of the moments to this parameter. For large $\sigma_\eta$ the observation of a charge is less likely to influence the observation of a second charge, which is similar to having a lower efficiency. For low efficiencies one expects the behavior to behave more like a Poissonian, and that $C_4/C_2$ and $C_3/C_1$ would be be closer to unity. Indeed this is the case.

Finally, we compare to STAR data. Figure \ref{fig:star} provides two comparisons. In the upper panel, each sub-volume has the same fixed charge, with the net electric charge being set to half the net baryon charge. The ratios $C_4/C_2$ and $C_3/C_2$ are both below the experimental results. The difference for net charge is especially striking. In the lower panel, the same procedure is followed with one exception. Rather than keeping the net charge fixed at the same value for each sub-volume, the baryon number and electric charge are assigned values according to Poissonian distributions, where the means are chosen to match the same fixed values as applied in the upper panel. Physically, this picture is based on having two distinct sources of charged particles. The first source comes from baryons and charge that originated from the target and projectile, and were moved into the mid-rapidity region. Given that those charges are displaced by several units of rapidity, their charge balance is likely accounted for well outside the range of the measurement, and it is not unreasonable to think of their contribution to each sub-volume as being random. The second source of charge would be that coming from strings, radiation, and decays. The charges from these processes are not able to transport large distances, hence the phenomenological picture based on sub-volumes. As expected, the ratios displayed in the lower panel are higher than those of the upper panel, for which charges were fixed to the same value for each sub-volume.

For net-baryon fluctuations, where the ratio $C_4/C_2$ from STAR is mainly below unity, it seems that charge conservation is the main source of non-Poissonian fluctuation. The experimental results seem higher, but statistical uncertainties make it difficult to make strong conclusions. In contrast, the net-charge fluctuations are far above the calculations of this paper, with the ratios roughly a factor of two higher than the calculations. It is difficult to see what missing physics might explain the discrepancy. If large volume fluctuations were responsbible, one would expect the baryon fluctuations to also differ more substantially. As seen in Sec. \ref{sec:uniformalpha}, Bose effects for pions would be unlikely to change the result more than 10\%. Phase separation would be expected to lead to significant fluctuations in baryon number, and it might be responsible for some of the model/data mismatch of the baryon fluctuations. However, it is more difficult to see how phase separation would lead to larger net-charge fluctuations than net-baryon fluctuations.

A principal impediment to interpreting the STAR results is in understanding the role of volume fluctuations. This could be better constrained by simultaneously investigating multiplicity fluctuations with the centrality bins and kinematic cuts used for net-charge and net-baryon fluctuations. Net-charge and net-baryon fluctuations should also not be investigated independently, as adjusting parameters to explain net-baryon number behavior while not reproducing net-charge fluctuations suggests that important physical considerations are still missing from the modeling. 








