% !TEX root =  CCmoments.tex


\section{Generation of Uncorrelated Sample Events}\label{sec:theoryMC}

Complicated experimental acceptances are difficult to incorporate into expressions for the moments. It is then easiest to generate entire events via Monte Carlo, and filter the events through the acceptance. The Monte Carlo procedure involves choosing a hadron proportional the number of ways the system might have such hadron, i.e. a product of the partition function of the individual hadron multiplied by the partition function of the remainder. The procedure becomes:
\begin{enumerate}
\item Calculate and store the partition function, $Z_A(Q,B,S)$, up to some size $A\le A_{\rm max}$ for all $Q,B,S$ that might ultimately couple back to a given $A=A_{\rm max}$ for the given total values $Q,B,S$. 
\item For total charge $Q,B,S$, choose the number of hadrons $A$ proportional to $Z_A(Q,B,S)/Z_A(Q,B,S)$.
\item Choose a hadron $h$ proportional to the probability $z_hZ_{A-1}(Q-q_h,B-b_h,S-s_h)/Z_A(Q,B,S)$. If Bose degeneracy is to be taken into account this procedure is slight modified as described in Sec. \ref{sec:bose}.
\item Choose the momentum proportional to the thermal weight $e^{-E_p/T}$.
\item Repeat (3,4) but with $A,Q,B,S$ being replaced by $A-1,Q-q_h,B-b-h,S-s_h$. The procedure is finished when $A=0$.
\end{enumerate}
Bose effects for pions can be included by altering the second and third steps above. In addition to choosing a hadron, one might also consider adding a $2-$pion, $3-$pion or $n-$pion state. Simply treat the $n-$pion state as if it were a resonance with charges $nq_h,nq_b,nq_s$, but with a partition function $z_{h,n}$ described above. If one chooses to make such a ``particle'', the $n$ identical particles are are given the same momentum and spin projection, using a Boltzmann distribution with a temperature $T/n$. To be more realistic the momenta should be smeared by some momenta $\approx\hbar/R$, where $R$ is a characteristic size of the system. If Fermi effects were to be included, one could calculate the partition function by adding a factor $(-1)^{n-1}$ to each term in the sum described in Eq. (\ref{eq:bose}). However, because some of the weights are negative, Monte Carlo procedures can become problematic. Fortunately, for the systems considered here are extremely hot, and degeneracy of fermions is negligible.

Storing the partition function can require substantial memory for larger $A_{\rm max}$ because the indices $Q,B$ and $S$ must also vary over a range of order $\pm A_{\rm max}$, so memory usage roughly scales with $A_{\rm max}^4$. Because one is usually interested in calculations with total charge near zero, one can ignore partition functions for charges that cannot couple back to the fixed overall charge at $A_{\rm max}$. Once $A$ exceeds $A_{\rm max}/2$, the calculations here cutoff values of $Q,B$ and $S$ that could not ultimately affect the $Q=B=S=0$ partition function for $A=A_{\rm max}$. Even with this savings, partition functions with $A_{\rm max}=250$ could require approximately 13 GB of memory, and need on the order of 10 minutes to calculate on a single processor. For $A_{\rm max}=125$, less than a GB of memory was needed and partition functions could be calculated in less than a minute. For hadron gases at temperatures of 150 MeV, $A_{\rm max}=250$ was readily sufficient for patch volumes $\lesssim 700$ fm$^3$. If multiple patch volumes are to be explored for the same temperature, computational time can also be saved by realizing that the partition functions scale as $\Omega^A$. Thus, if one performs a calculation for some initial volume $\Omega_0$, scaling can provide results for new volumes with minimal computation.

Once the partition function is calculated, event generation is remarkably fast. The time to generate an even scales linearly with the volume, or equivalently, linear with the average number of particles generated. Running sufficient events to generate a million individual particles can be accomplished within a few seconds on a single CPU. Unlike Metropolis methods where events are modified by considering small changes to existing events, such as in \cite{}, each event in this is independent as long as the random numbers are without correlation.
