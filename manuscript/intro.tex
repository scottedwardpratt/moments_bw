% !TEX root =  CCmoments.tex

\section{Introduction}\label{sec:intro}

The fluctuation of conserved charges is a standard means by which to investigate and classify phase transitions. At the critical point correlation lengths diverge, which results in peaks in charge fluctuations as one approaches the critical point. For systems with first-order phase transitions, fluctuations become infinite whenever one is in a region of phase separation, as the liquid and gas regions prefer to become infinite in extent when the surface energy is non-zero. The growth of fluctuations becomes increasingly dramatic as one considers progressively higher-order fluctuations. In a volume $V$, fluctuations of a charge $Q$ can be defined as
\begin{eqnarray}\label{eq:kappadef}
\mathcal{M}_N&\equiv&\frac{1}{V}\langle(Q-\overline{Q})^N\rangle=\frac{1}{V}\sum_n P_n(n-\overline{n})^N,
\end{eqnarray}
when particles have unit charge. The measure is increasingly sensitive to the tails of the multiplicity distribution, $P_n$, as $N$ increases. The free energy, $F\sim a_2(n-\overline{n})^2 + a_3(n-\overline{n})^3\cdots$, is minimized for $n=\overline{n}$, but the quadratic part vanishes at the critical point, $a_2\rightarrow 0$, which allow the fluctuations to grow and be dominated by higher-order terms. The shape of the tails of the distribution can be profoundly altered. 

The properties of the QCD  transition, deconfinement and the restoration of chiral symmetry, are not well understood at finite baryon density. There exists the possibility that this transition is a true phase transition, with a critical point at several times nuclear density and with a critical temperature close to the pion mass. If this is the case, it begs the question as to whether the conditions for phase separation or for critical phenomena can be reproduced in the laboratory. Heavy ion collisions at high energy, measured at the Relativistic Heavy Ion Collider (RHIC) or a the LHC, can produce mesoscopic regions a temperatures of a few hundred MeV, which is well above the expectations for a critical temperature, and densities of several times nuclear matter density. 

High-energy heavy ion collisions are characterized by strong explosive collective flow. Measurement is confined to the outgoing asymptotic momenta, but because of strong flow, correlations in coordinate space manifest themselves as correlations in relative momentum. Thus, measurements of correlations binned by relative momentum or charge fluctuations within some defined region of momentum space serve as surrogates for the corresponding observables in coordinate space. Indeed, measurements of charge and baryon number fluctuations have been performed at RHIC. By adjusting the beam energy of the colliding nuclei, experiments at RHIC have explored conditions at which novel phase phenomena might occur. Fluctuations of electric charge and baryon number have been especially popular. An initial scan of beam energies was rather inconclusive, but measurements with greatly improved statistics are currently being analyzed. 

In addition to the finite system size (event multiplicities might number in the thousands), the novel states of matter created in heavy-ion collisions persist for $\lesssim 10$ fm/$c$. This severely limits the degree to which phases can separate or to which critical fluctuations can grow. From the initial overlap of the colliding nuclei, until the mean time at which emitted particles experience their last collision, the collision lasts a few tens of fm/$c$. This limits the degree to which conserved charges can separate from one another. For example if a strange and an anti-strange quanta are produced together, their separation is limited by diffusion, and fluctuations of strangeness, or any other charge, requires that charges be given sufficient time to leave and enter the defining volume.

The goal of this paper is to gauge the degree to which charge-balance correlations affect higher-order correlations. In addition to making it difficult for phases to separate or for critical correlations to emerge, conservation also represents its own source of correlation, which needs to be understood as a potential sourcs of background before making firm arguments to have observed phenomena related to phase transitions. It is well known that charge-balance correlations are readily measurable at the two particle level, $N=2$ in Eq. (\ref{eq:kappadef}), and that they explain the bulk of the $N=2$ fluctuation measurement. However, their impact on $N=3,4$ fluctuations has not be investigated in great detail. For instance, the four-particle measure of correlation,
\begin{eqnarray}
C_4&=&\mathcal{M}_4-3\mathcal{M}_2^2,
\end{eqnarray}
which is based on cumulants, subtracts much of the contribution to the $\mathcal{M}_4$ coming from purely two-particle correlations. However, charge conservation can involve multiple particles, and the degree to which a cumulant-based measure, like the kurtosis, is affected by charge conservation is not fully understood. Relations based on a uniform acceptance probability and for a single type of charge were worked out in \cite{Savchuk:2019xfg}, which provide significant insight into how higher-order correlations are affected by charge conservation. The goal of this study is to extend such ideas to a more realistic picture, which takes into account the conservation of all three types of charge (baryon number, electric charge and strangeness) and applies a more realistic model of experimental acceptance and efficiency. The interplays of charge conservation with chemical equilibrium, decays and Bose statistics are all considered.

To understand the role of chemical equilibrium and decays, a model is presented which creates small volumes in which the net charge, $B,Q$ and $S$, is fixed at some value. Even if the net charges are all zero, charged particles exist but in combinations that conserve the net charge. Theoretical methods for exact calculation of the canonical ensemble and a method for Monte Carlo generation of statistically independent events are presented here. The method lends itself to including decays and accounting for experimental acceptance and efficiency. The physical picture of treating small volumes as independent patches was previously done for calculation of charge balance functions in \cite{Schlichting:2010qia,Schlichting:2010na}, and was also applied in \cite{Oliinychenko:2020cmr}. Following the terminology in \cite{Oliinychenko:2020cmr}, we refer to these sub-volumes as patches. The method presented here creates perfectly independent samplings, and can generate billions of such patches within a few hours. This enables highly accurate calculations of high moments with minimal numeric costs.

The next section reviews the definition of cumulants, skewness and kurtosis. Theoretical methods are presented in the next section. This includes a review of how one can exactly calculate canonical ensembles for multiple charges in a hadron gas and independently generate sets of particles for a given sub-volume, or patch.  Simplified expressions for a single type of charge with a fixed efficiency, which were derived and presented in \cite{Savchuk:2019xfg}, are also listed here as they provide a strong base from which to understand the behavior of more complicated treatments.

After a brief review of cumulants and the definitions of skewness and kurtosis in Sec. \ref{sec:cumulaants}, the method for exact solution of the canonical ensemble describing a multi-component, multi-charge hadron gas is presented in Sec. \ref{sec:theory}. These techniques extend those used for canonical ensembles used to study isospin fluctuations of a hadron gas \cite{}, nuclear fragmentation \cite{}, the level density of a Fermi gas, and the effect of restricting a quark-gluon plasma (QGP) to having fixed charge, including being in an overall color singlet \cite{}. Exact methods for calculating correlations up to fourth-order are presented. Unfortunately, to include realistic acceptance effects and complex decays, the exact expressions are no longer viable. However, as shown in Sec. \ref{sec:theoryMC}, the exact expressions show how sample events can be generated. An event, defined by a set of particles and momenta, are generated with perfect independence from one another, and perfectly reproduce the canonical expressions from Sec. \ref{sec:theoryexact}. Section \ref{sec:theorybosefermi} extends the previous sections to show how Bose correlations can be included. Section \ref{sec:uniformeff} considers the case of a single type of charge and a uniform efficiency, i.e. the probability of any particle being recorded is set to some fixed value. Much of this discussion repeats what is said in \cite{Savchuck:2019xfg}, and is included for completeness. This provides for a physical discussion of how charge conservation, volume fluctuations, chemical equilibrium, decays, clustering, and Bose corrections should affect higher moments.

The heart of this study is presented in Sec. \ref{sec:blastwave}. Here, the patches are assigned collective velocities consistent with the collective flow deduced from heavy-ion collisions. A canonical sampling of particles are generated from each patch, followed by a simulation of their decays. The particles were then overlaid onto the acceptance of the STAR detector at RHIC. Each patch is uncorrelated with any other patch, so moments can be calculated by averaging over the independent contributions from the patches. Results are displayed alongside results from the STAR Collaboration. The size and sign of the fluctuations of the net-proton distributions are consistent with observations, but the calculations of the net charge distributions differs qualitatively from STAR observations. A detailed discussion of the lessons derived from this study is presented in Sec. \ref{sec:summary}.
