\documentclass[letterpaper,aps,showpacs,floatfix,10pt,prc]{revtex4-1}
\usepackage{graphicx}
\usepackage{amsmath}
\usepackage{graphicx}
\usepackage{comment}
\usepackage{float}

\begin{document}
\title{Just some notes at this point}
\author{Scott Pratt}
\affiliation{Department of Physics and Astronomy and National Superconducting Cyclotron Laboratory\\
Michigan State University, East Lansing, MI 48824~~USA}
\author{Dmitro Oliinchenko}
\affiliation{Nuclear Science Division, Lawrence Berkeley National Laboratory, Berkeley, CA, 94720, USA}
\date{\today}

\pacs{}

\begin{abstract}
blah blah
\end{abstract}

\maketitle

\section{Iterative Calculation of Canonical Partition Function}

We consider the canonical partition function for independent hadrons where $B$, $Q$ and $S$ are all conserved integral quantities, $Z(B,Q,S)$. We add an additional index, $N$, referencing the number of hadrons.
\begin{eqnarray}
\label{eq:Zsum}
Z(B,Q,S)&=&\sum_N Z_N(B,Q,S).
\end{eqnarray}
One can write $Z_N$ as
\begin{eqnarray}
Z_N(B,Q,S)&=&\sum_{\langle \vec{n} \rm{~s.t.}\sum n_i=N\rangle}\prod_i\frac{z_i^{n_i}}{n_i!}\delta[\sum_in_ib_i-B]\delta[\sum_in_iq_i-Q]\delta[\sum_in_is_i-S]\delta[\sum_in_i-N].
\end{eqnarray}
Now, to derive the iterative relation. First, insert $(\sum_i n_i/N)$, which is unity, into the expression. Then, rewrite,
\begin{eqnarray}\label{eq:iterativeN}
\nonumber
Z_N(B,Q,S)&=&\sum_{\langle \vec{n} \rm{~s.t.}\sum n_i=N\rangle}\prod_i\frac{z_i^{n_i}}{n_i!}
\left(\frac{\sum_jn_j}{N}\right)
\delta[\sum_in_ib_i-B]\delta[\sum_in_iq_i-Q]\delta[\sum_in_is_i-S]\delta[\sum_in_i-N]\\
\nonumber
&=&\frac{1}{N}\sum_j z_j\sum_{\langle \vec{n} \rm{~s.t.}\sum n_i=N\rangle}\prod_i\frac{z_i^{n_i}}{n_i!}\left(\frac{n_j}{z_j}\right)
\delta[\sum_in_ib_i-B]\delta[\sum_in_iq_i-Q]\delta[\sum_in_is_i-S]\delta[\sum_in_i-N]\\
\nonumber
&=&\frac{1}{N}\sum_j z_j
\sum_{\langle \vec{n} \rm{~s.t.}\sum n_i=N-1\rangle}\prod_i\frac{z_i^{n_i}}{n_i!}
\delta[\sum_in_ib_i-B-b_j]\delta[\sum_in_iq_i-Q-q_j]\delta[\sum_in_is_i-S-s_j]
\delta[\sum_in_i-N-1]\\
&=&\frac{1}{N}\sum_jz_jZ_{N-1}(B-b_j,Q-q_j,S-s_j).
\end{eqnarray}
Thus, beginning with $Z_0(0,0,0)=1$, one can find all $Z_N(B,Q,S)$ iteratively.

Equation (\ref{eq:iterativeN}) is the main result of this section. It provides a mechanism for calculating $Z(B,Q,S)$, but only after first calculating $Z_N(B,Q,S)$, then summing over $N$, see Eq. (\ref{eq:Zsum}), to obtain the actual partition function. A natural question is whether one can design an iterative procedure without first calculating as a function of $N$. Adding this index increases the dimensionality of the calculation, which increases both the numerical cost and the amount of memory for calculating $Z(B,Q,S)$. Indeed, one can derive recursion relations using any of the three charges, $B,Q$ or $S$, and forego any mention of $N$. For example, one can write a relation with respect to $B$ using the analogous steps above,
\begin{equation}
\label{eq:iterateB}
Z(B,Q,S)=\frac{1}{B}\sum_i b_iz_i Z(B-b_i,Q-q_i,S-s_i).
\end{equation}
Unfortunately, in order for this procedure to iterate from $Z(0,0,0)$ it relies on the label, in this case $B$, being only positive. For non-relativistic physics, this is not a problem, however for relativistic physics $B$ can be both positive and negative, which destroys the chance of treating this as an iterative procedure. Even if one only considers baryons, $Z(0,0,0)\ne 1$ because the system might be composed of some number of mesons, which would require an iterative procedure be applied using $Q$ and $S$ and $N_{\rm{mesons}}$ as indices to calculate $Z(0,0,0)$.

\section{Higher Moments and Fluctuations}
One can generate quantities of the form, $\langle n_in_j\cdots\rangle$, rather easily once one has $Z_N(B,Q,S)$. For example,
\begin{eqnarray}label{eq:hadfluc}
\langle n_in_jn_k\rangle&=&
\frac{1}{Z(B,Q,S)}
\sum_N \sum_{\langle \vec{n} \rm{~s.t.}\sum n_i=N\rangle}
\prod_i\frac{z_\ell^{n_\ell}}{n_\ell!}n_in_jn_k\\
\nonumber
&&\cdot\delta[\sum_\ell n_\ell b_\ell-B]\delta[\sum_\ell n_\ell q_\ell-Q]\delta[\sum_\ell n_is_\ell-S]
\delta[\sum_\ell n_\ell-N]\\
\nonumber
&=&
\frac{z_iz_jz_k}{Z(B,Q,S)}
\sum_N \sum_{\langle \vec{n} \rm{~s.t.}\sum n_i=N-3\rangle}
\prod_i\frac{z_\ell^{n_\ell}}{n_\ell!}n_in_jn_k\\
\nonumber
&&\cdot\delta[\sum_\ell n_\ell b_\ell-B-b_i-b_j-b_k]
\delta[\sum_\ell n_\ell q_\ell-Q-q_i-q_j-q_k]
\delta[\sum_\ell n_is_\ell-S-s_i-s_j-s_k]
\delta[\sum_\ell n_\ell-N]\\
\nonumber
&=&\frac{z_iz_jz_k}{Z(B,Q,S)}
\sum_N Z_{N-3}(B-b_i-b_j-b_k,Q-q_i-q_j-q_k,S-s_i-s_j-s_k)\\
\nonumber
&=&\frac{z_iz_jz_k}{Z(B,Q,S)}
Z(B-b_i-b_j-b_k,Q-q_i-q_j-q_k,S-s_i-s_j-s_k).
\end{eqnarray}
Expressions for fluctuations indexed by hadron species then become straightforward.

Equation \ref{eq:hadfluc} can be extended to the calculation of charge fluctuation, by writing $Q=\sum_i q_in_i$.  Of course, charges in a canonical ensemble don't fluctuate, but it is a good thing to check. For example, From Eq. (\ref{eq:hadfluc}) the quantity 
\begin{eqnarray}
\label{eq:Q2step1}
\langle Q^2\rangle&=&
\sum_i\sum_j \frac{q_iz_iq_jz_j}{Z(B,Q,S)}Z(B-b_i-b_j,Q-q_iq_j,S-s_i-s_j).
\end{eqnarray}
Using Eq. (\ref{eq:iterateB}), but with $Q$ serving as the iterative variable,
\begin{eqnarray}
Z(B,Q,S)&=&\sum\frac{q_iz_i}{Q}Z(B-b_i,Q-q_i,S-s_i),
\end{eqnarray}
one can eliminate the terms with the index $i$ from Eq. (\ref{eq:Q2step1}) and get
\begin{eqnarray}
\langle Q^2\rangle&=&Q
\sum_j\frac{q_jz_j}{Z(B,Q,S)}Z(B-b_j,Q-q_j,S-s_j).
\end{eqnarray}
Again using Eq. (\ref{eq:iterateB}), one finds the desired result,
\begin{eqnarray}
\langle Q^2\rangle=Q^2,
\end{eqnarray}
which shows that
\begin{eqnarray}
\langle(Q-\langle Q\rangle)^2\rangle &=&\langle Q^2\rangle-\langle Q\rangle^2
=Q^2-Q^2=0.
\end{eqnarray}
Similarly, as expected all charge fluctuations involving powers of the charges $B-\langle B\rangle$, $Q-\langle Q\rangle$ and $S-\langle S\rangle$ can be shown to vanish.

\section{Generating sets of particles via Monte Carlo}

Here, we describe how one creates a group of particles at temperature $T$ in a finites volume $V$ with fixed $B$, $Q$ and $S$, using the partition function $Z_N(B,Q,S)$, presented earlier. The first step is to choose $N$, proportional to the probability,
\begin{eqnarray}
P_N=\frac{Z_N(B,Q,S)}{Z(B,Q,S)}.
\end{eqnarray}
Using Eq. (\ref{eq:Zsum}), it is clear that $\sum_N P_N=1$ for any $B,Q,S$. 

Once $N$ is chosen, one can imagine having $N$ hadrons in the gas, then randomly picking one hadron. The chance one picks a species $i$ is
\begin{eqnarray}
p_i&=&\frac{\langle n_i\rangle}{N}\\
\nonumber
&=&z_i\frac{Z_{N-1}(B-b_i,Q-Q_i,S-S_i)}{Z_N(B,Q,S)},
\end{eqnarray}
which sums to unity. The system is now in the new set of states, $N-1,B-b_1,Q-q_1,S-s_1$, where the subscript ``$1$'' identifies the charge of the first particle. Because the first particle was chosen randomly, the remainder of the system is also randomly, or thermally, populated, and the second particle can be chosen in the same manner,
\begin{eqnarray}
p_j&=&\frac{\langle n_n\rangle}{N}\\
\nonumber
&=&z_j\frac{Z_{N-2}(B-b_i,Q-Q_i,S-S_i)}{Z_{N-1}(B-b_1,Q-q_1,S-s_1)}.
\end{eqnarray}
One then simply repeats the procedure until $N$ hadrons have been chosen to finish the generation of one event. 

Evaluating a system where the average number of hadrons is $\langle N\rangle\sim 100$ requires one to generate and store $Z_N(B,Q,S)$ up to some maximum number $N_{\rm max}$. A cutoff $\gtrsim 200$ seems to be sufficient as the system tends to fall off with characteristic scale $\sqrt{\langle N\rangle}$. Calculating these partition functions can take $\sim 5$ minutes on a single CPU, and the time increases proportional to $N_{\rm max}^4$. Thus, this method may not be reasonable for a system of sizes $\gtrsim 1000$. 

In order to increase the efficiency, one may neglect any combination $B,Q,S$ that is not reachable via the decay from an original system with $N_{\rm max}$ hadrons. For example, if one is considering a system with $B=Q=S=0$, there is no reason to calculate for combinations where $B>N_{\rm max}-N$. Once a specific value of $B$ is being considered, one can calculate the cutoff for electric charge, because there is only one hadron with more than unit charge, the $\Delta^{++}$. Given $B$ and $Q$ one can calculate the cutoffs for $S$ given that the only multiply strange hadrons are baryons of specific charge. A detailed criteria of cutoffs can reduce the storage by a factor of 8.

If one were to ignore anti-baryons, and work with $Z(B,Q,S)$, it would be difficult to deal with the fact that the zero-baryon case holds many particles. Thus, if one wished to produce samples of hadrons, one might still prefer to retain the index $N$ even though one could calculate the partition function, integrated over $N$, without actually referencing $N$.

\section{Multi-bosons}

One could include Bose effects of pions by treating the case where $n_+$ positive pions, $n_-$ negative pions and $n_0$ neutral pions were all in a specific single particle level as a single state of charge $+n_+$, $-n_-$ or zero respectively. When calculating $z_i$ for this species, one would simply calculate $z$ of one pion species, but evaluated as a function of $T/n_{+,-,0}$. For instance, one could calculate to third order by adding resonances representing $1,2$ and 3-pion states. Thus, instead of three pion resonances, there would be 9. When one would Monte-Carlo the ``resonance'' one would put all the pions into the same single-particle state. This is straight-forward, but would increase the storage by a factor of $\approx 3$, because the 5-pion state would have an electric charge of $+5,-5,0$, requiring the partition function to be calculated for a much larger range of $Q$ for a given $N$. Only pions are significantly affected by Bose effects at the temperatures and densities expected at the RHIC energy scan. At very low temperatures, $T\lesssim 50$ MeV, Fermi effects of baryons can play a role. This can easily be included when calculating the partition function to a given order of the degeneracy. The difference being that the $2,4,6\cdots$ degeneracies contribute a negative weight, $z_i<0$. Negative weights might be problematic for the Monte Carlo generation of multi-particle states.  

%\begin{acknowledgments}
%This work was supported by the Department of Energy Office of Science through grant number DE-FG02-03ER41259.
%\end{acknowledgments}

%% !TEX root =  CCmoments.tex

\begin{thebibliography}{99}

%\cite{Abdelwahab:2014yha}
\bibitem{Abdelwahab:2014yha}
N.~Abdelwahab \textit{et al.} [STAR],
%``Energy Dependence of $K/\pi$, $p/\pi$, and $K/p$ Fluctuations in Au+Au Collisions from $\rm \sqrt{s_{NN}}$ = 7.7 to 200 GeV,''
Phys. Rev. C \textbf{92}, no.2, 021901 (2015)
doi:10.1103/PhysRevC.92.021901
[arXiv:1410.5375 [nucl-ex]].
%20 citations counted in INSPIRE as of 23 Jun 2020

%\cite{Sarkar:2014fja}
\bibitem{Sarkar:2014fja}
A.~Sarkar [STAR],
%``Higher-moment measurements of net-kaon, net-charge and net-proton multiplicity distributions at STAR,''
Nucl. Phys. A \textbf{931}, 796-801 (2014)
doi:10.1016/j.nuclphysa.2014.10.042
%1 citations counted in INSPIRE as of 23 Jun 2020

%\cite{Xu:2016hxf}
\bibitem{Xu:2016hxf}
J.~Xu [STAR],
%``Higher Moments of Net-Kaon Multiplicity Distributions at STAR,''
J. Phys. Conf. Ser. \textbf{779}, no.1, 012073 (2017)
doi:10.1088/1742-6596/779/1/012073
[arXiv:1611.07132 [hep-ex]].
%4 citations counted in INSPIRE as of 23 Jun 2020

%\cite{Thader:2016gpa}
\bibitem{Thader:2016gpa}
J.~Thäder [STAR],
%``Higher Moments of Net-Particle Multiplicity Distributions,''
Nucl. Phys. A \textbf{956}, 320-323 (2016)
doi:10.1016/j.nuclphysa.2016.02.047
[arXiv:1601.00951 [nucl-ex]].
%41 citations counted in INSPIRE as of 23 Jun 2020

%\cite{Collaboration):2013qla}
\bibitem{Collaboration):2013qla}
A.~Sarkar [STAR],
%``Higher moments of net kaon multiplicity distributions at RHIC energies for the search of QCD Critical Point at STAR,''
EPJ Web Conf. \textbf{60}, 13013 (2013)
doi:10.1051/epjconf/20136013013
%0 citations counted in INSPIRE as of 23 Jun 2020

%\cite{Luo:2013saa}
\bibitem{Luo:2013saa}
X.~Luo [STAR],
%``Beam Energy Dependence of Higher Moments of Net-proton Multiplicity Distributions in Heavy-ion Collisions at RHIC,''
PoS \textbf{CPOD2013}, 019 (2013)
doi:10.22323/1.185.0019
[arXiv:1306.3106 [nucl-ex]].
%11 citations counted in INSPIRE as of 23 Jun 2020

%\cite{Sahoo:2012pfx}
\bibitem{Sahoo:2012pfx}
N.~R.~Sahoo [STAR],
%``Probing the QCD critical point by higher moments of the net-charge and net-proton multiplicity distribution in STAR Experiment,''
DAE Symp. Nucl. Phys. \textbf{57}, 766-767 (2012)
%0 citations counted in INSPIRE as of 23 Jun 2020

%\cite{Luo:2012kja}
\bibitem{Luo:2012kja}
X.~Luo [STAR],
%``Search for the QCD Critical Point by Higher Moments of Net-proton Multiplicity Distributions at STAR,''
Nucl. Phys. A \textbf{904-905}, 911c-914c (2013)
doi:10.1016/j.nuclphysa.2013.02.163
[arXiv:1210.5573 [nucl-ex]].
%37 citations counted in INSPIRE as of 23 Jun 2020

%\cite{Tarnowsky:2011vk}
\bibitem{Tarnowsky:2011vk}
T.~J.~Tarnowsky [STAR],
%``Searching for the QCD Critical Point Using Particle Ratio Fluctuations and Higher Moments of Multiplicity Distributions,''
J. Phys. G \textbf{38}, 124054 (2011)
doi:10.1088/0954-3899/38/12/124054
[arXiv:1106.6110 [nucl-ex]].
%25 citations counted in INSPIRE as of 23 Jun 2020

%\cite{Luo:2011ts}
\bibitem{Luo:2011ts}
X.~F.~Luo [STAR],
%``Probing the QCD Critical Point with Higher Moments of Net-proton Multiplicity Distributions,''
J. Phys. Conf. Ser. \textbf{316}, 012003 (2011)
doi:10.1088/1742-6596/316/1/012003
[arXiv:1106.2926 [nucl-ex]].
%55 citations counted in INSPIRE as of 23 Jun 2020

%\cite{Sahoo:2011at}
\bibitem{Sahoo:2011at}
N.~R.~Sahoo [STAR],
%``Probing the QCD Critical Point by Higher Moments of Net-Charge Distribution,''
[arXiv:1101.5125 [nucl-ex]].
%2 citations counted in INSPIRE as of 23 Jun 2020

%\cite{R.Sahoo:2011qtk}
\bibitem{R.Sahoo:2011qtk}
N.~R.~Sahoo [STAR],
%``Search for the QCD critical point by the higher moments of the net-charge and net-proton multiplicity distribution in STAR Experiment at RHIC,''
DAE Symp. Nucl. Phys. \textbf{56}, 1028-1029 (2011)
%0 citations counted in INSPIRE as of 23 Jun 2020

%\cite{Aggarwal:2010wy}
\bibitem{Aggarwal:2010wy}
M.~Aggarwal \textit{et al.} [STAR],
%``Higher Moments of Net-proton Multiplicity Distributions at RHIC,''
Phys. Rev. Lett. \textbf{105}, 022302 (2010)
doi:10.1103/PhysRevLett.105.022302
[arXiv:1004.4959 [nucl-ex]].
%316 citations counted in INSPIRE as of 23 Jun 2020

%\cite{Chatterjee:2019msq}
\bibitem{Chatterjee:2019msq}
A.~Chatterjee [STAR],
%``Off-diagonal cumulants of net-charge, net-proton and net-kaon multiplicity distributions in Au+Au collisions at $\sqrt{s_{NN}}$ = 7.7-200 GeV from STAR,''
PoS \textbf{CORFU2018}, 164 (2019)
doi:10.22323/1.347.0164
[arXiv:1904.01302 [hep-ex]].
%0 citations counted in INSPIRE as of 23 Jun 2020

%\cite{Adam:2019xmk}
\bibitem{Adam:2019xmk}
J.~Adam \textit{et al.} [STAR],
%``Collision-energy dependence of second-order off-diagonal and diagonal cumulants of net-charge, net-proton, and net-kaon multiplicity distributions in Au + Au collisions,''
Phys. Rev. C \textbf{100}, no.1, 014902 (2019)
doi:10.1103/PhysRevC.100.014902
[arXiv:1903.05370 [nucl-ex]].
%14 citations counted in INSPIRE as of 23 Jun 2020

%\cite{Adamczyk:2014fia}
\bibitem{Adamczyk:2014fia}
L.~Adamczyk \textit{et al.} [STAR],
%``Beam energy dependence of moments of the net-charge multiplicity distributions in Au+Au collisions at RHIC,''
Phys. Rev. Lett. \textbf{113}, 092301 (2014)
doi:10.1103/PhysRevLett.113.092301
[arXiv:1402.1558 [nucl-ex]].
%286 citations counted in INSPIRE as of 23 Jun 2020

%\cite{McDonald:2012ts}
\bibitem{McDonald:2012ts}
D.~McDonald [STAR],
%``Beam energy and centrality dependence of the statistical moments of the net-charge and net-kaon multiplicity distributions in Au+Au collisions at STAR,''
Nucl. Phys. A \textbf{904-905}, 907c-910c (2013)
doi:10.1016/j.nuclphysa.2013.02.162
[arXiv:1210.7023 [nucl-ex]].
%16 citations counted in INSPIRE as of 23 Jun 2020

%\cite{McDonald:2013aoa}
\bibitem{McDonald:2013aoa}
D.~McDonald,
%``Statistical moments of the multiplicity distributions of identified particles in Au+Au collisions,''
%0 citations counted in INSPIRE as of 23 Jun 2020
Ph.D. thesis, Rice University, \href{https://drupal.star.bnl.gov/STAR/files/mcdonaldthesis.pdf}{https://drupal.star.bnl.gov/STAR/files/mcdonaldthesis.pdf} (2013).

%\cite{Bzdak:2012ab}
\bibitem{Bzdak:2012ab}
A.~Bzdak and V.~Koch,
%``Acceptance corrections to net baryon and net charge cumulants,''
Phys. Rev. C \textbf{86}, 044904 (2012)
doi:10.1103/PhysRevC.86.044904
[arXiv:1206.4286 [nucl-th]].
%114 citations counted in INSPIRE as of 06 Jul 2020

%\cite{Cheng:2002jb}
\bibitem{Cheng:2002jb}
S.~Cheng and S.~Pratt,
%``Isospin fluctuations from a thermally equilibrated hadron gas,''
Phys. Rev. C \textbf{67}, 044904 (2003)
doi:10.1103/PhysRevC.67.044904
[arXiv:nucl-th/0207051 [nucl-th]].
%4 citations counted in INSPIRE as of 06 Jul 2020

%\cite{Pratt:2003jd}
\bibitem{Pratt:2003jd}
S.~Pratt and J.~Ruppert,
%``The Quark gluon plasma in a finite volume,''
Phys. Rev. C \textbf{68}, 024904 (2003)
doi:10.1103/PhysRevC.68.024904
[arXiv:nucl-th/0303043 [nucl-th]].
%6 citations counted in INSPIRE as of 06 Jul 2020

%\cite{Pratt:1999ns}
\bibitem{Pratt:1999ns}
S.~Pratt,
%``Canonical and microcanonical calculations for Fermi systems,''
Phys. Rev. Lett. \textbf{84}, 4255-4259 (2000)
doi:10.1103/PhysRevLett.84.4255
[arXiv:nucl-th/9905055 [nucl-th]].
%13 citations counted in INSPIRE as of 06 Jul 2020

%\cite{Pratt:1999ht}
\bibitem{Pratt:1999ht}
S.~Pratt and S.~Das Gupta,
%``Statistical models of nuclear fragmentation,''
Phys. Rev. C \textbf{62}, 044603 (2000)
doi:10.1103/PhysRevC.62.044603
[arXiv:nucl-th/9903006 [nucl-th]].
%25 citations counted in INSPIRE as of 06 Jul 2020

%\cite{Carruthers:1989jj}
\bibitem{Carruthers:1989jj}
P.~Carruthers, E.~M.~Friedlander, C.~C.~Shih and R.~M.~Weiner,
%``Multiplicity Fluctuations in Finite Rapidity Windows: Intermittency or Quantum Statistical Correlation?,''
Phys. Lett. B \textbf{222}, 487-492 (1989)
doi:10.1016/0370-2693(89)90350-X
%102 citations counted in INSPIRE as of 06 Jul 2020

%\cite{Carruthers:1983my}
\bibitem{Carruthers:1983my}
P.~Carruthers and C.~C.~Shih,
%``Correlations and Fluctuations in Hadronic Multiplicity Distributions: The Meaning of KNO Scaling,''
Phys. Lett. B \textbf{127}, 242-250 (1983)
doi:10.1016/0370-2693(83)90884-5
%205 citations counted in INSPIRE as of 06 Jul 2020

%\cite{Das:2012yq}
\bibitem{Das:2012yq}
S.~Das [STAR],
%``Centrality dependence of freeze-out parameters from the beam energy scan at STAR,''
Nucl. Phys. A \textbf{904-905}, 891c-894c (2013)
doi:10.1016/j.nuclphysa.2013.02.158
[arXiv:1210.6099 [nucl-ex]].
%28 citations counted in INSPIRE as of 06 Jul 2020

%\cite{Kumar:2012np}
\bibitem{Kumar:2012np}
L.~Kumar [STAR],
%``Centrality dependence of freeze-out parameters from Au+Au collisions at $\sqrt{s_{NN}}=$7.7, 11.5 and 39 GeV,''
Central Eur. J. Phys. \textbf{10}, 1274-1277 (2012)
doi:10.2478/s11534-012-0097-9
[arXiv:1201.4203 [nucl-ex]].
%12 citations counted in INSPIRE as of 06 Jul 2020

%\cite{Abelev:2009bw}
\bibitem{Abelev:2009bw}
B.~I.~Abelev \textit{et al.} [STAR],
%``Identified particle production, azimuthal anisotropy, and interferometry measurements in Au+Au collisions at s(NN)**(1/2) = 9.2- GeV,''
Phys. Rev. C \textbf{81}, 024911 (2010)
doi:10.1103/PhysRevC.81.024911
[arXiv:0909.4131 [nucl-ex]].
%207 citations counted in INSPIRE as of 07 Jul 2020

%\cite{Yang:2017llt}
\bibitem{Yang:2017llt}
C.~Yang [STAR],
%``The STAR beam energy scan phase II physics and upgrades,''
Nucl. Phys. A \textbf{967}, 800-803 (2017)
doi:10.1016/j.nuclphysa.2017.05.042
%24 citations counted in INSPIRE as of 20 Jul 2020

%\cite{Pratt:2018ebf}
\bibitem{Pratt:2018ebf}
S.~Pratt and C.~Plumberg,
%``Evolving Charge Correlations in a Hybrid Model with both Hydrodynamics and Hadronic Boltzmann Descriptions,''
Phys. Rev. C \textbf{99}, no.4, 044916 (2019)
doi:10.1103/PhysRevC.99.044916
[arXiv:1812.05649 [nucl-th]].
%4 citations counted in INSPIRE as of 20 Jul 2020

%\cite{Pratt:2019pnd}
\bibitem{Pratt:2019pnd}
S.~Pratt and C.~Plumberg,
%``Determining the Diffusivity for Light Quarks from Experiment,''
[arXiv:1904.11459 [nucl-th]].
%3 citations counted in INSPIRE as of 20 Jul 2020

%\cite{Pratt:2019fbj}
\bibitem{Pratt:2019fbj}
S.~Pratt,
%``Calculating $n$-Point Charge Correlations in Evolving Systems,''
Phys. Rev. C \textbf{101}, no.1, 014914 (2020)
doi:10.1103/PhysRevC.101.014914
[arXiv:1908.01053 [nucl-th]].
%0 citations counted in INSPIRE as of 20 Jul 2020








%\cite{Schlichting:2010qia}
\bibitem{Schlichting:2010qia}
S.~Schlichting and S.~Pratt,
%``Charge conservation at energies available at the BNL Relativistic Heavy Ion Collider and contributions to local parity violation observables,''
Phys. Rev. C \textbf{83} (2011), 014913
doi:10.1103/PhysRevC.83.014913
[arXiv:1009.4283 [nucl-th]].
%149 citations counted in INSPIRE as of 19 May 2020

%\cite{Pratt:2010zn}
\bibitem{Pratt:2010zn}
S.~Pratt, S.~Schlichting and S.~Gavin,
%``Effects of Momentum Conservation and Flow on Angular Correlations at RHIC,''
Phys. Rev. C \textbf{84} (2011), 024909
doi:10.1103/PhysRevC.84.024909
[arXiv:1011.6053 [nucl-th]].
%95 citations counted in INSPIRE as of 19 May 2020

%\cite{Schlichting:2010na}
\bibitem{Schlichting:2010na}
S.~Schlichting and S.~Pratt,
%``Effects of Charge Conservation and Flow on Fluctuations of parity-odd Observables ar RHIC,''
[arXiv:1005.5341 [nucl-th]].
%28 citations counted in INSPIRE as of 19 May 2020

%\cite{Oliinychenko:2020cmr}
\bibitem{Oliinychenko:2020cmr}
D.~Oliinychenko, S.~Shi and V.~Koch,
%``Effects of local event-by-event conservation laws in ultra-relativistic heavy ion collisions at the particlization,''
[arXiv:2001.08176 [hep-ph]].
%0 citations counted in INSPIRE as of 19 May 2020

%\cite{Tanabashi:2018oca}
\bibitem{Tanabashi:2018oca}
M.~Tanabashi \textit{et al.} [Particle Data Group],
%``Review of Particle Physics,''
Phys. Rev. D \textbf{98} (2018) no.3, 030001
doi:10.1103/PhysRevD.98.030001
%4561 citations counted in INSPIRE as of 13 May 2020

%\cite{Savchuk:2019xfg}
\bibitem{Savchuk:2019xfg}
O.~Savchuk, R.~V.~Poberezhnyuk, V.~Vovchenko and M.~I.~Gorenstein,
%``Binomial acceptance corrections for particle number distributions in high-energy reactions,''
Phys. Rev. C \textbf{101} (2020) no.2, 024917
doi:10.1103/PhysRevC.101.024917
[arXiv:1911.03426 [hep-ph]].
%2 citations counted in INSPIRE as of 05 May 2020

%\cite{Oliinychenko:2020cmr}
\bibitem{Oliinychenko:2020cmr} 
  D.~Oliinychenko, S.~Shi and V.~Koch,
  %``Effects of local event-by-event conservation laws in ultra-relativistic heavy ion collisions at the particlization,''
  arXiv:2001.08176 [hep-ph].
  %%CITATION = ARXIV:2001.08176;%%
  
  %\cite{Pratt:1999ht}
\bibitem{Pratt:1999ht} 
  S.~Pratt and S.~Das Gupta,
  %``Statistical models of nuclear fragmentation,''
  Phys.\ Rev.\ C {\bf 62}, 044603 (2000)
  doi:10.1103/PhysRevC.62.044603
  [nucl-th/9903006].
  %%CITATION = doi:10.1103/PhysRevC.62.044603;%%
  %25 citations counted in INSPIRE as of 29 Mar 2020

%\cite{Cheng:2002jb}
\bibitem{Cheng:2002jb}
S.~Cheng and S.~Pratt,
%``Isospin fluctuations from a thermally equilibrated hadron gas,''
Phys.\ Rev.\ C \textbf{67}, 044904 (2003)
doi:10.1103/PhysRevC.67.044904
[arXiv:nucl-th/0207051 [nucl-th]].
%4 citations counted in INSPIRE as of 30 Mar 2020

\bibitem{Pratt:2003jd}
S.~Pratt and J.~Ruppert,
%``The Quark gluon plasma in a finite volume,''
Phys.\ Rev.\ C \textbf{68}, 024904 (2003)
doi:10.1103/PhysRevC.68.024904
[arXiv:nucl-th/0303043 [nucl-th]].
%6 citations counted in INSPIRE as of 30 Mar 2020

%\cite{Sangaline:2015bma}
\bibitem{Sangaline:2015bma}
E.~Sangaline,
%``Strongly Intensive Cumulants: Fluctuation Measures for Systems With Incompletely Constrained Volumes,''
[arXiv:1505.00261 [nucl-th]].
%15 citations counted in INSPIRE as of 20 May 2020

%\cite{Gorenstein:2011vq}
\bibitem{Gorenstein:2011vq}
M.~Gorenstein and M.~Gazdzicki,
%``Strongly Intensive Quantities,''
Phys. Rev. C \textbf{84} (2011), 014904
doi:10.1103/PhysRevC.84.014904
[arXiv:1101.4865 [nucl-th]].
%96 citations counted in INSPIRE as of 20 May 2020

%\cite{Gazdzicki:2013ana}
\bibitem{Gazdzicki:2013ana}
M.~Gazdzicki, M.~Gorenstein and M.~Mackowiak-Pawlowska,
%``Normalization of strongly intensive quantities,''
Phys. Rev. C \textbf{88} (2013) no.2, 024907
doi:10.1103/PhysRevC.88.024907
[arXiv:1303.0871 [nucl-th]].
%49 citations counted in INSPIRE as of 20 May 2020

%\cite{Begun:2014boa}
\bibitem{Begun:2014boa}
V.~V.~Begun, M.~I.~Gorenstein and K.~Grebieszkow,
%``Strongly Intensive Measures for Particle Number Fluctuations: Effects of Hadronic Resonances,''
J. Phys. G \textbf{42} (2015) no.7, 075101
doi:10.1088/0954-3899/42/7/075101
[arXiv:1409.3023 [nucl-th]].
%4 citations counted in INSPIRE as of 20 May 2020

%\cite{Greiner:1993jn}
\bibitem{Greiner:1993jn}
C.~Greiner, C.~Gong and B.~Muller,
%``Some remarks on pion condensation in relativistic heavy ion collisions,''
Phys. Lett. B \textbf{316} (1993), 226-230
doi:10.1016/0370-2693(93)90317-B
[arXiv:hep-ph/9307336 [hep-ph]].
%67 citations counted in INSPIRE as of 13 May 2020

%\cite{Pratt:1993uy}
\bibitem{Pratt:1993uy}
S.~Pratt,
%``Pion lasers from high-energy collisions,''
Phys. Lett. B \textbf{301} (1993), 159-164
doi:10.1016/0370-2693(93)90682-8
%136 citations counted in INSPIRE as of 13 May 2020

%\cite{Bzdak:2018uhv}
\bibitem{Bzdak:2018uhv}
A.~Bzdak, V.~Koch, D.~Oliinychenko and J.~Steinheimer,
%``Large proton cumulants from the superposition of ordinary multiplicity distributions,''
Phys. Rev. C \textbf{98} (2018) no.5, 054901
doi:10.1103/PhysRevC.98.054901
[arXiv:1804.04463 [nucl-th]].
%11 citations counted in INSPIRE as of 13 May 2020

%\cite{Flores:2016mtp}
\bibitem{Flores:2016mtp}
C.~E.~Flores [STAR],
%``The Rapidity Density Distributions and Longitudinal Expansion Dynamics of Identified Pions from the STAR Beam Energy Scan,''
Nucl. Phys. A \textbf{956} (2016), 280-283
doi:10.1016/j.nuclphysa.2016.05.020
%1 citations counted in INSPIRE as of 04 Jun 2020

%\cite{Stephanov:1998dy}
\bibitem{Stephanov:1998dy}
M.~A.~Stephanov, K.~Rajagopal and E.~V.~Shuryak,
%``Signatures of the tricritical point in QCD,''
Phys. Rev. Lett. \textbf{81}, 4816-4819 (1998)
doi:10.1103/PhysRevLett.81.4816
[arXiv:hep-ph/9806219 [hep-ph]].
%985 citations counted in INSPIRE as of 06 Jul 2020

%\cite{Stephanov:1999zu}
\bibitem{Stephanov:1999zu}
M.~A.~Stephanov, K.~Rajagopal and E.~V.~Shuryak,
%``Event-by-event fluctuations in heavy ion collisions and the QCD critical point,''
Phys. Rev. D \textbf{60}, 114028 (1999)
doi:10.1103/PhysRevD.60.114028
[arXiv:hep-ph/9903292 [hep-ph]].
%811 citations counted in INSPIRE as of 06 Jul 2020

\bibitem{Bass:2000az} 
  S.~A.~Bass, P.~Danielewicz and S.~Pratt,
  %``Clocking hadronization in relativistic heavy ion collisions with balance functions,''
  Phys.\ Rev.\ Lett.\  {\bf 85}, 2689 (2000).







%\cite{Pratt:2015zsa}
\bibitem{Pratt:2015zsa} 
  S.~Pratt, E.~Sangaline, P.~Sorensen and H.~Wang,
  %``Constraining the Eq. of State of Super-Hadronic Matter from Heavy-Ion Collisions,''
  Phys.\ Rev.\ Lett.\  {\bf 114}, 202301 (2015).
  %doi:10.1103/PhysRevLett.114.202301.
  %[arXiv:1501.04042 [nucl-th]].
  %%CITATION = doi:10.1103/PhysRevLett.114.202301;%%
  %62 citations counted in INSPIRE as of 21 Mar 2019
  
	
%\cite{Pratt:2015jsa}
\bibitem{Pratt:2015jsa} 
  S.~Pratt, W.~P.~McCormack and C.~Ratti,
  %``Production of Charge in Heavy Ion Collisions,''
  Phys.\ Rev.\ C {\bf 92}, 064905 (2015).
  %doi:10.1103/PhysRevC.92.064905
  %[arXiv:1508.07031 [nucl-th]].
  %%CITATION = doi:10.1103/PhysRevC.92.064905;%%
  %4 citations counted in INSPIRE as of 19 Dec 2017 
  
%\cite{Bernhard:2016tnd}
\bibitem{Bernhard:2016tnd} 
  J.~E.~Bernhard, J.~S.~Moreland, S.~A.~Bass, J.~Liu and U.~Heinz,
  %``Applying Bayesian parameter estimation to relativistic heavy-ion collisions: simultaneous characterization of the initial state and quark-gluon plasma medium,''
  Phys.\ Rev.\ C {\bf 94}, no. 2, 024907 (2016).
  %doi:10.1103/PhysRevC.94.024907
 % [arXiv:1605.03954 [nucl-th]].
  %%CITATION = doi:10.1103/PhysRevC.94.024907;%%
  %150 citations counted in INSPIRE as of 21 Mar 2019
  
%\cite{Bernhard:2015hxa}
\bibitem{Bernhard:2015hxa} 
  J.~E.~Bernhard, P.~W.~Marcy, C.~E.~Coleman-Smith, S.~Huzurbazar, R.~L.~Wolpert and S.~A.~Bass,
  %``Quantifying properties of hot and dense QCD matter through systematic model-to-data comparison,''
  Phys.\ Rev.\ C {\bf 91}, no. 5, 054910 (2015).
  %doi:10.1103/PhysRevC.91.054910
 % [arXiv:1502.00339 [nucl-th]].
  %%CITATION = doi:10.1103/PhysRevC.91.054910;%%
  %36 citations counted in INSPIRE as of 21 Mar 2019

%\cite{Burke:2013yra}
\bibitem{Burke:2013yra}
  K.~M.~Burke {\it et al.} [JET Collaboration],
  %``Extracting the jet transport coefficient from jet quenching in high-energy heavy-ion collisions,''
  Phys.\ Rev.\ C {\bf 90}, no. 1, 014909 (2014).
 % doi:10.1103/PhysRevC.90.014909
 % [arXiv:1312.5003 [nucl-th]].
  %%CITATION = doi:10.1103/PhysRevC.90.014909;%%
  %215 citations counted in INSPIRE as of 21 Mar 2019

%\cite{He:2018gks}
\bibitem{He:2018gks} 
  Y.~He, L.~G.~Pang and X.~N.~Wang,
  %``Bayesian extraction of jet energy loss distributions in heavy-ion collisions,''
  arXiv:1808.05310 [hep-ph].
  %%CITATION = ARXIV:1808.05310;%%
  %3 citations counted in INSPIRE as of 21 Mar 2019

%\cite{Xu:2017obm}
\bibitem{Xu:2017obm} 
  Y.~Xu, J.~E.~Bernhard, S.~A.~Bass, M.~Nahrgang and S.~Cao,
  %``Data-driven analysis for the temperature and momentum dependence of the heavy-quark diffusion coefficient in relativistic heavy-ion collisions,''
  Phys.\ Rev.\ C {\bf 97}, no. 1, 014907 (2018).
 % doi:10.1103/PhysRevC.97.014907
 % [arXiv:1710.00807 [nucl-th]].
  %%CITATION = doi:10.1103/PhysRevC.97.014907;%%
  %27 citations counted in INSPIRE as of 21 Mar 2019
  
\bibitem{Greif:2017byw} 
  M.~Greif, J.~A.~Fotakis, G.~S.~Denicol and C.~Greiner,
  Phys.\ Rev.\ Lett.\  {\bf 120}, 242301 (2018).

\bibitem{Pratt:2019fbj} 
  S.~Pratt,
  %``Calculating n-Point Charge Correlations in Evolving Systems,''
  arXiv:1908.01053 [nucl-th].

\bibitem{Borsanyi:2011sw}
  S.~Borsanyi, Z.~Fodor, S.~D.~Katz, S.~Krieg, C.~Ratti and K.~Szabo,
  %``Fluctuations of conserved charges at finite temperature from lattice QCD,''
  JHEP {\bf 1201}, 138 (2012).
 % [arXiv:1112.4416 [hep-lat]].
  %%CITATION = ARXIV:1112.4416;%%

%\cite{Aarts:2014nba}
\bibitem{Aarts:2014nba} 
  G.~Aarts, C.~Allton, A.~Amato, P.~Giudice, S.~Hands and J.~I.~Skullerud,
  %``Electrical conductivity and charge diffusion in thermal QCD from the lattice,''
  JHEP {\bf 1502}, 186 (2015).
  %doi:10.1007/JHEP02(2015)186
  %[arXiv:1412.6411 [hep-lat]].
  %%CITATION = doi:10.1007/JHEP02(2015)186;%%
  %71 citations counted in INSPIRE as of 27 Nov 2017

%\cite{Amato:2013naa}
\bibitem{Amato:2013naa} 
  A.~Amato, G.~Aarts, C.~Allton, P.~Giudice, S.~Hands and J.~I.~Skullerud,
  %``Electrical conductivity of the quark-gluon plasma across the deconfinement transition,''
  Phys.\ Rev.\ Lett.\  {\bf 111}, no. 17, 172001 (2013).
 % doi:10.1103/PhysRevLett.111.172001
 % [arXiv:1307.6763 [hep-lat]].
  %%CITATION = doi:10.1103/PhysRevLett.111.172001;%%
  %119 citations counted in INSPIRE as of 25 Mar 2019

%\cite{Policastro:2002se} diffusion from AdS-CFT
\bibitem{Policastro:2002se} 
  G.~Policastro, D.~T.~Son and A.~O.~Starinets,
  %``From AdS / CFT correspondence to hydrodynamics,''
  JHEP {\bf 0209}, 043 (2002).
 % doi:10.1088/1126-6708/2002/09/043
 % [hep-th/0205052].
  %%CITATION = doi:10.1088/1126-6708/2002/09/043;%%
  %641 citations counted in INSPIRE as of 25 Mar 2019

%\cite{CasalderreySolana:2006rq} D for heavy quarks from AdS/CFT
\bibitem{CasalderreySolana:2006rq} 
  J.~Casalderrey-Solana and D.~Teaney,
  %``Heavy quark diffusion in strongly coupled N=4 Yang-Mills,''
  Phys.\ Rev.\ D {\bf 74}, 085012 (2006).
  %doi:10.1103/PhysRevD.74.085012
  %[hep-ph/0605199].
  %%CITATION = doi:10.1103/PhysRevD.74.085012;%%
  %383 citations counted in INSPIRE as of 25 Mar 2019

%\cite{Ghiglieri:2018dib}
\bibitem{Ghiglieri:2018dib} 
  J.~Ghiglieri, G.~D.~Moore and D.~Teaney,
  %``QCD Shear Viscosity at (almost) NLO,''
  JHEP {\bf 1803}, 179 (2018).
  %doi:10.1007/JHEP03(2018)179
  %[arXiv:1802.09535 [hep-ph]].
  %%CITATION = doi:10.1007/JHEP03(2018)179;%%
  %13 citations counted in INSPIRE as of 01 May 2019
  
  
\bibitem{Greif:2016skc} 
  M.~Greif, C.~Greiner and G.~S.~Denicol,
  %``Electric conductivity of a hot hadron gas from a kinetic approach,''
  Phys.\ Rev.\ D {\bf 93}, no. 9, 096012 (2016)
  Erratum: [Phys.\ Rev.\ D {\bf 96}, no. 5, 059902 (2017)]
  
\bibitem{Hammelmann:2018ath} 
  J.~Hammelmann, J.~M.~Torres-Rincon, J.~B.~Rose, M.~Greif and H.~Elfner,
  %``Electrical conductivity and relaxation via colored noise in a hadronic gas,''
  Phys.\ Rev.\ D {\bf 99}, no. 7, 076015 (2019).

%\cite{Pratt:2018ebf}
\bibitem{Pratt:2018ebf} 
  S.~Pratt and C.~Plumberg,
  %``Evolving Charge Correlations in a Hybrid Model with both Hydrodynamics and Hadronic Boltzmann Descriptions,''
  to appear in Phys. Rev. C, arXiv:1812.05649 [nucl-th].
  %%CITATION = ARXIV:1812.05649;%%
  %1 citations counted in INSPIRE as of 11 Apr 2019

\bibitem{Shen:2014vra} 
  C.~Shen, Z.~Qiu, H.~Song, J.~Bernhard, S.~Bass and U.~Heinz,
  %``The iEBE-VISHNU code package for relativistic heavy-ion collisions,''
  Comput.\ Phys.\ Commun.\  {\bf 199}, 61 (2016)

%\cite{Pratt:2017oyf}
\bibitem{Pratt:2017oyf}
  S.~Pratt, J.~Kim and C.~Plumberg,
  %``Evolution of Charge Fluctuations and Correlations in the Hydrodynamic Stage of Heavy Ion Collisions,''
  Phys.\ Rev.\ C {\bf 98}, no. 1, 014904 (2018).
  %doi:10.1103/PhysRevC.98.014904.
%  [arXiv:1712.09298 [nucl-th]].
  %%CITATION = doi:10.1103/PhysRevC.98.014904;%%

%\cite{Wang:2012jua}
\bibitem{Wang:2012jua} %% Hui thesis
  H.~Wang, Ph.D. Thesis,
  %``Study of particle ratio fluctuations and charge balance functions at RHIC,''
  arXiv:1304.2073 [nucl-ex].
  %%CITATION = ARXIV:1304.2073;%%
  %4 citations counted in INSPIRE as of 29 Oct 2017

%\cite{Abelev:2010ab}
\bibitem{Abelev:2010ab} 
  B.~I.~Abelev {\it et al.} [STAR Collaboration],
  %``Longitudinal scaling property of the charge balance function in Au + Au collisions at 200 GeV,''
  Phys.\ Lett.\ B {\bf 690}, 239 (2010).
  %doi:10.1016/j.physletb.2010.05.028
  %[arXiv:1002.1641 [nucl-ex]].
  %%CITATION = doi:10.1016/j.physletb.2010.05.028;%%
  %10 citations counted in INSPIRE as of 14 Dec 2017

%\cite{Li:2011zzx}
%\bibitem{Li:2011zzx} 
  N.~Li {\it et al.} [STAR Collaboration],
  %``The study of longitudinal properties of the charge balance function,''
%  Indian J.\ Phys.\  {\bf 85}, 923 (2011).
  %doi:10.1007/s12648-011-0100-0
  %%CITATION = doi:10.1007/s12648-011-0100-0;%%

%\cite{Adams:2003kg}
\bibitem{Adams:2003kg} 
  J.~Adams {\it et al.} [STAR Collaboration],
  %``Narrowing of the balance function with centrality in au + au collisions at (S(NN))**1/2 = 130-GeV,''
  Phys.\ Rev.\ Lett.\  {\bf 90}, 172301 (2003).
  %doi:10.1103/PhysRevLett.90.172301
  %[nucl-ex/0301014].
  %%CITATION = doi:10.1103/PhysRevLett.90.172301;%%
  %88 citations counted in INSPIRE as of 14 Dec 2017 
   
%\cite{Aggarwal:2010ya}
\bibitem{Aggarwal:2010ya} 
  M.~M.~Aggarwal {\it et al.} [STAR Collaboration],
  %``Balance Functions from Au$+$Au, $d+$Au, and $p+p$ Collisions at $\sqrt{s_{NN}}$ = 200 GeV,''
  Phys.\ Rev.\ C {\bf 82}, 024905 (2010).
  %doi:10.1103/PhysRevC.82.024905
  %[arXiv:1005.2307 [nucl-ex]].
  %%CITATION = doi:10.1103/PhysRevC.82.024905;%%
  %43 citations counted in INSPIRE as of 14 Dec 2017

%\cite{Alt:2004gx}
%\bibitem{Alt:2004gx} 
 % C.~Alt {\it et al.} [NA49 Collaboration],
  %``System size and centrality dependence of the balance function in A + A collisions at s(NN)**(1/2) = 17.2-GeV,''
 % Phys.\ Rev.\ C {\bf 71}, 034903 (2005).
  %doi:10.1103/PhysRevC.71.034903
  %[hep-ex/0409031].
  %%CITATION = doi:10.1103/PhysRevC.71.034903;%%
  %36 citations counted in INSPIRE as of 14 Dec 2017

%\cite{Abelev:2013csa}
\bibitem{Abelev:2013csa}
  B.~Abelev {\it et al.} [ALICE Collaboration],
  %``Charge correlations using the balance function in Pb-Pb collisions at $\sqrt{s_{NN}}$ = 2.76 TeV,''
  Phys.\ Lett.\ B {\bf 723}, 267 (2013).
  %doi:10.1016/j.physletb.2013.05.039
  %[arXiv:1301.3756 [nucl-ex]].
  %%CITATION = doi:10.1016/j.physletb.2013.05.039;%%
  %22 citations counted in INSPIRE as of 14 Dec 2017
  %measured both for Delta phi and Delta eta
  
%\cite{Alt:2007hk}
\bibitem{Alt:2007hk} 
  C.~Alt {\it et al.} [NA49 Collaboration],
  %``Rapidity and energy dependence of the electric charge correlations in A + A collisions at the SPS energies,''
  Phys.\ Rev.\ C {\bf 76}, 024914 (2007).
  %doi:10.1103/PhysRevC.76.024914
  %[arXiv:0705.1122 [nucl-ex]].
  %%CITATION = doi:10.1103/PhysRevC.76.024914;%%
  %19 citations counted in INSPIRE as of 14 Dec 2017

%\cite{Adamczyk:2015yga}
\bibitem{Adamczyk:2015yga} 
  L.~Adamczyk {\it et al.} [STAR Collaboration],
  %``Beam-energy dependence of charge balance functions from Au + Au collisions at energies available at the BNL Relativistic Heavy Ion Collider,''
  Phys.\ Rev.\ C {\bf 94}, no. 2, 024909 (2016).
  %doi:10.1103/PhysRevC.94.024909
  %[arXiv:1507.03539 [nucl-ex]].
  %%CITATION = doi:10.1103/PhysRevC.94.024909;%%
  %3 citations counted in INSPIRE as of 29 Oct 2017

%\cite{Adamczyk:2013hsi}
\bibitem{Adamczyk:2013hsi} 
  L.~Adamczyk {\it et al.} [STAR Collaboration],
  %``Fluctuations of charge separation  perpendicular to the event plane and local parity violation in $\sqrt{s_{NN}}=200$ GeV Au+Au  collisions at the BNL Relativistic Heavy Ion Collider,''
  Phys.\ Rev.\ C {\bf 88}, no. 6, 064911 (2013).
 % doi:10.1103/PhysRevC.88.064911.
%  [arXiv:1302.3802 [nucl-ex]].
  %%CITATION = doi:10.1103/PhysRevC.88.064911;%%
  %57 citations counted in INSPIRE as of 20 Nov 2018

%\cite{Abelev:2009ac}
\bibitem{Abelev:2009ac} 
  B.~I.~Abelev {\it et al.} [STAR Collaboration],
  %``Azimuthal Charged-Particle Correlations and Possible Local Strong Parity Violation,''
  Phys.\ Rev.\ Lett.\  {\bf 103}, 251601 (2009).
 % doi:10.1103/PhysRevLett.103.251601.
%  [arXiv:0909.1739 [nucl-ex]].
  %%CITATION = doi:10.1103/PhysRevLett.103.251601;%%
  %397 citations counted in INSPIRE as of 20 Nov 2018


%\cite{Pratt:2012dz}
\bibitem{Pratt:2012dz} 
  S.~Pratt,
  %``Identifying the Charge Carriers of the Quark-Gluon Plasma,''
  Phys.\ Rev.\ Lett.\  {\bf 108}, 212301 (2012).
  %doi:10.1103/PhysRevLett.108.212301
  %[arXiv:1203.4578 [nucl-th]].
  %%CITATION = doi:10.1103/PhysRevLett.108.212301;%%
  %22 citations counted in INSPIRE as of 19 Dec 2017


%\cite{Pan:2014caa}
\bibitem{Pan:2014caa} 
  Y.~Pan and S.~Pratt,
  %``Baryon annihilation and regeneration in heavy ion collisions,''
  Phys.\ Rev.\ C {\bf 89}, no. 4, 044911 (2014).
  %doi:10.1103/PhysRevC.89.044911
  %%CITATION = doi:10.1103/PhysRevC.89.044911;%%
  %10 citations counted in INSPIRE as of 19 Dec 2017

%\cite{Steinheimer:2017vju}
\bibitem{Steinheimer:2017vju} 
  J.~Steinheimer, J.~Aichelin, M.~Bleicher and H.~Stöcker,
  %``Influence of the hadronic phase on observables in ultrarelativistic heavy ion collisions,''
  Phys.\ Rev.\ C {\bf 95}, no. 6, 064902 (2017).
  %doi:10.1103/PhysRevC.95.064902
  %[arXiv:1703.06638 [nucl-th]].
  %%CITATION = doi:10.1103/PhysRevC.95.064902;%%
  %5 citations counted in INSPIRE as of 19 Dec 2017
 
%\cite{Steinheimer:2012rd}
\bibitem{Steinheimer:2012rd} 
  J.~Steinheimer, J.~Aichelin and M.~Bleicher,
  %``Nonthermal p/? Ratio at LHC as a Consequence of Hadronic Final State Interactions,''
  Phys.\ Rev.\ Lett.\  {\bf 110}, no. 4, 042501 (2013).
  %doi:10.1103/PhysRevLett.110.042501
  %[arXiv:1203.5302 [nucl-th]].
  %%CITATION = doi:10.1103/PhysRevLett.110.042501;%%
  %84 citations counted in INSPIRE as of 19 Dec 2017
  
%%%%%%%%%%%%%%%%%%%%%%% XXXXXXX

\begin{comment}

%\cite{Bellwied:2015lba}
%\bibitem{Bellwied:2015lba} 
  R.~Bellwied, S.~Borsanyi, Z.~Fodor, S.~D.~Katz, A.~Pasztor, C.~Ratti and K.~K.~Szabo,
  %``Fluctuations and correlations in high temperature QCD,''
  Phys.\ Rev.\ D {\bf 92}, no. 11, 114505 (2015).
 % doi:10.1103/PhysRevD.92.114505.
%  [arXiv:1507.04627 [hep-lat]].
  %%CITATION = doi:10.1103/PhysRevD.92.114505;%%
  %74 citations counted in INSPIRE as of 28 Mar 2018


%\bibitem{Ling:2013ksb} 
  B.~Ling, T.~Springer and M.~Stephanov,
  %``Hydrodynamics of charge fluctuations and balance functions,''
  Phys.\ Rev.\ C {\bf 89}, no. 6, 064901 (2014).

%\cite{Bozek:2004dt}
%\bibitem{Bozek:2004dt} 
  P.~Bozek,
  %``The Balance functions in azimuthal angle is a measure of the transverse flow,''
  Phys.\ Lett.\ B {\bf 609}, 247 (2005).
  %doi:10.1016/j.physletb.2005.01.072
  %[nucl-th/0412076].
  %%CITATION = doi:10.1016/j.physletb.2005.01.072;%%
  %29 citations counted in INSPIRE as of 19 Dec 2017


%\cite{Cheng:2004zy}
%\bibitem{Cheng:2004zy} 
  S.~Cheng, S.~Petriconi, S.~Pratt, M.~Skoby, C.~Gale, S.~Jeon, V.~Topor Pop and Q.~H.~Zhang,
  %``Statistical and dynamic models of charge balance functions,''
  Phys.\ Rev.\ C {\bf 69}, 054906 (2004).
  %doi:10.1103/PhysRevC.69.054906
  %[nucl-th/0401008].
  %%CITATION = doi:10.1103/PhysRevC.69.054906;%%
  %35 citations counted in INSPIRE as of 19 Dec 2017

%\bibitem{popcorn}
	P. Ed\'en and G. Gustafson, Zeit. f\"ur Phys. C, {\bf 75}, 41 (1997).

%\bibitem{urqmd}
	S.A.Bass et al [URQMD], Progr. Part. Nucl. Physics Vol. \textbf{41}, 225 (1998).

%\cite{Alt:2007hk}
%\bibitem{Alt:2007hk} 
  C.~Alt {\it et al.} [NA49 Collaboration],
  %``Rapidity and energy dependence of the electric charge correlations in A + A collisions at the SPS energies,''
  Phys.\ Rev.\ C {\bf 76}, 024914 (2007).
  %doi:10.1103/PhysRevC.76.024914
  %[arXiv:0705.1122 [nucl-ex]].
  %%CITATION = doi:10.1103/PhysRevC.76.024914;%%
  %19 citations counted in INSPIRE as of 14 Dec 2017

%\cite{Adamczyk:2015yga}
%\bibitem{Adamczyk:2015yga} 
  L.~Adamczyk {\it et al.} [STAR Collaboration],
  %``Beam-energy dependence of charge balance functions from Au + Au collisions at energies available at the BNL Relativistic Heavy Ion Collider,''
  Phys.\ Rev.\ C {\bf 94}, no. 2, 024909 (2016).
  %doi:10.1103/PhysRevC.94.024909
  %[arXiv:1507.03539 [nucl-ex]].
  %%CITATION = doi:10.1103/PhysRevC.94.024909;%%
  %3 citations counted in INSPIRE as of 29 Oct 2017

%\cite{Pan:2015pzh}
%\bibitem{Pan:2015pzh} 
  Y.~H.~Pan and W.~N.~Zhang,
  %``Charge balance functions in a scenario of continuing charge production in quark matter,''
  Eur.\ Phys.\ J.\ A {\bf 51}, no. 11, 147 (2015).
  %doi:10.1140/epja/i2015-15147-3
  %%CITATION = doi:10.1140/epja/i2015-15147-3;%%

%\bibitem{Heinz:2013wva} 
  U.~W.~Heinz,
  %``Towards the Little Bang Standard Model,''
  J.\ Phys.\ Conf.\ Ser.\  {\bf 455}, 012044 (2013).

%\bibitem{sorgepionwind}
	H. Sorge, Phys. Lett. B 373, 16 􏰞1993􏰀.
	
%\cite{Pratt:1998gt}
%\bibitem{Pratt:1998gt} 
  S.~Pratt and J.~Murray,
  %``Modeling the breakup stage of relativistic heavy ion collisions,''
  Phys.\ Rev.\ C {\bf 57}, 1907 (1998).
  %doi:10.1103/PhysRevC.57.1907
  %%CITATION = doi:10.1103/PhysRevC.57.1907;%%
  %23 citations counted in INSPIRE as of 27 Nov 2018
 

%\cite{Novak:2013bqa}
%\bibitem{Novak:2013bqa} 
  J.~Novak, K.~Novak, S.~Pratt, J.~Vredevoogd, C.~Coleman-Smith and R.~Wolpert,
  %``Determining Fundamental Properties of Matter Created in Ultrarelativistic Heavy-Ion Collisions,''
  Phys.\ Rev.\ C {\bf 89}, no. 3, 034917 (2014).
 % doi:10.1103/PhysRevC.89.034917.
 % [arXiv:1303.5769 [nucl-th]].
  %%CITATION = doi:10.1103/PhysRevC.89.034917;%%
  %61 citations counted in INSPIRE as of 29 Nov 2018 

%\cite{Nayak:2009wc}
\bibitem{Nayak:2009wc}
T.~K.~Nayak [STAR],
%``Study of the Fluctuations of Net-charge and Net-protons Using Higher Order Moments,''
Nucl. Phys. A \textbf{830}, 555C-558C (2009)
doi:10.1016/j.nuclphysa.2009.09.046
[arXiv:0907.4542 [nucl-ex]].
%26 citations counted in INSPIRE as of 23 Jun 2020

%\cite{Abelev:2008jg}
\bibitem{Abelev:2008jg}
B.~Abelev \textit{et al.} [STAR],
%``Beam-Energy and System-Size Dependence of Dynamical Net Charge Fluctuations,''
Phys. Rev. C \textbf{79}, 024906 (2009)
doi:10.1103/PhysRevC.79.024906
[arXiv:0807.3269 [nucl-ex]].
%58 citations counted in INSPIRE as of 23 Jun 2020








%\cite{Shen:2014vra}
%\bibitem{Shen:2014vra} 
  C.~Shen, Z.~Qiu, H.~Song, J.~Bernhard, S.~Bass and U.~Heinz,
  %``The iEBE-VISHNU code package for relativistic heavy-ion collisions,''
  Comput.\ Phys.\ Commun.\  {\bf 199}, 61 (2016).

%\bibitem{Huovinen:2009yb} 
  P.~Huovinen and P.~Petreczky,
  %``QCD Equation of State and Hadron Resonance Gas,''
  Nucl.\ Phys.\ A {\bf 837}, 26 (2010).

%\bibitem{Kapusta:2014dja} 
  J.~I.~Kapusta and C.~Young,
  %``Causal Baryon Diffusion and Colored Noise,''
  Phys.\ Rev.\ C {\bf 90}, no. 4, 044902 (2014).

%\cite{Kapusta:2017hfi}
%\bibitem{Kapusta:2017hfi} 
  J.~I.~Kapusta and C.~Plumberg,
  %``Causal Electric Charge Diffusion and Balance Functions in Relativistic Heavy Ion Collisions,''
  Phys.\ Rev.\ C {\bf 97}, no. 1, 014906 (2018).
 % doi:10.1103/PhysRevC.97.014906.
 % [arXiv:1710.03329 [nucl-th]].
  %%CITATION = doi:10.1103/PhysRevC.97.014906;%%
  %8 citations counted in INSPIRE as of 13 Dec 2018
  
%\cite{Aziz:2004qu}
%\bibitem{Aziz:2004qu} 
  M.~A.~Aziz and S.~Gavin,
  %``Causal diffusion and the survival of charge fluctuations in nuclear collisions,''
  Phys.\ Rev.\ C {\bf 70}, 034905 (2004).
%  doi:10.1103/PhysRevC.70.034905.
 % [nucl-th/0404058].
  %%CITATION = doi:10.1103/PhysRevC.70.034905;%%
  %58 citations counted in INSPIRE as of 28 Nov 2018

%\bibitem{Cooper:1974mv} 
  F.~Cooper and G.~Frye,
  %``Comment on the Single Particle Distribution in the Hydrodynamic and Statistical Thermodynamic Models of Multiparticle Production,''
  Phys.\ Rev.\ D {\bf 10}, 186 (1974).

%\bibitem{Huovinen:2012is} 
  P.~Huovinen and H.~Petersen,
  %``Particlization in hybrid models,''
  Eur.\ Phys.\ J.\ A {\bf 48}, 171 (2012).

%\cite{Becattini:2012sq}
%\bibitem{Becattini:2012sq} 
  F.~Becattini, M.~Bleicher, T.~Kollegger, M.~Mitrovski, T.~Schuster and R.~Stock,
  %``Hadronization and Hadronic Freeze-Out in Relativistic Nuclear Collisions,''
  Phys.\ Rev.\ C {\bf 85}, 044921 (2012).
  %doi:10.1103/PhysRevC.85.044921
  %[arXiv:1201.6349 [nucl-th]].
  %%CITATION = doi:10.1103/PhysRevC.85.044921;%%
  %59 citations counted in INSPIRE as of 19 Dec 2017

%\cite{Becattini:2012xb}
%\bibitem{Becattini:2012xb} 
  F.~Becattini, M.~Bleicher, T.~Kollegger, T.~Schuster, J.~Steinheimer and R.~Stock,
  %``Hadron Formation in Relativistic Nuclear Collisions and the QCD Phase Diagram,''
  Phys.\ Rev.\ Lett.\  {\bf 111}, 082302 (2013).
  %doi:10.1103/PhysRevLett.111.082302
  %[arXiv:1212.2431 [nucl-th]].
  %%CITATION = doi:10.1103/PhysRevLett.111.082302;%%
  %110 citations counted in INSPIRE as of 19 Dec 2017

%\cite{Pratt:2014vja}
%\bibitem{Pratt:2014vja} 
  S.~Pratt,
  %``Accounting for backflow in hydrodynamic-Boltzmann interfaces,''
  Phys.\ Rev.\ C {\bf 89}, no. 2, 024910 (2014).
  %doi:10.1103/PhysRevC.89.024910
  %[arXiv:1401.0316 [nucl-th]]
  %%CITATION = doi:10.1103/PhysRevC.89.024910;%%
  %8 citations counted in INSPIRE as of 19 Dec 2017
%
%\bibitem{Pratt:2010jt} 
  S.~Pratt and G.~Torrieri,
  %``Coupling Relativistic Viscous Hydrodynamics to Boltzmann Descriptions,''
  Phys.\ Rev.\ C {\bf 82}, 044901 (2010).

%\cite{Bozek:2003qi}
%\bibitem{Bozek:2003qi} 
  P.~Bozek, W.~Broniowski and W.~Florkowski,
  %``Balance functions in a thermal model with resonances,''
  Acta Phys.\ Hung.\ A {\bf 22}, 149 (2005).
  %doi:10.1556/APH.22.2005.1-2.15
  %[nucl-th/0310062].
  %%CITATION = doi:10.1556/APH.22.2005.1-2.15;%%
  %35 citations counted in INSPIRE as of 19 Dec 2017

%\bibitem{WestfallAcceptance}
% Routines for modeling the efficiency of the STAR detector were generously provided by Gary Westfall.

%\cite{Steinheimer:2012bn}
%\bibitem{Steinheimer:2012bn} 
  J.~Steinheimer, V.~Koch and M.~Bleicher,
  %``Hydrodynamics at large baryon densities: Understanding proton vs. anti-proton v_2 and other puzzles,''
  Phys.\ Rev.\ C {\bf 86}, 044903 (2012).
  %doi:10.1103/PhysRevC.86.044903
  %[arXiv:1207.2791 [nucl-th]].
  %%CITATION = doi:10.1103/PhysRevC.86.044903;%%
  %36 citations counted in INSPIRE as of 19 Dec 2017

%\cite{Pratt:2003gh}
%\bibitem{Pratt:2003gh} 
  S.~Pratt and S.~Cheng,
  %``Removing distortions from charge balance functions,''
  Phys.\ Rev.\ C {\bf 68}, 014907 (2003).
  %doi:10.1103/PhysRevC.68.014907.
  %[nucl-th/0303025].
  %%CITATION = doi:10.1103/PhysRevC.68.014907;%%
  %28 citations counted in INSPIRE as of 29 Nov 2018
 
%\cite{Schlichting:2010qia}
%\bibitem{Schlichting:2010qia} 
  S.~Schlichting and S.~Pratt,
  %``Charge conservation at energies available at the BNL Relativistic Heavy Ion Collider and contributions to local parity violation observables,''
  Phys.\ Rev.\ C {\bf 83}, 014913 (2011).
 % doi:10.1103/PhysRevC.83.014913.
 % [arXiv:1009.4283 [nucl-th]].
  %%CITATION = doi:10.1103/PhysRevC.83.014913;%%
  %120 citations counted in INSPIRE as of 30 Nov 2018

%\cite{Pratt:2010zn}
%\bibitem{Pratt:2010zn} 
  S.~Pratt, S.~Schlichting and S.~Gavin,
  %``Effects of Momentum Conservation and Flow on Angular Correlations at RHIC,''
  Phys.\ Rev.\ C {\bf 84}, 024909 (2011).
  %doi:10.1103/PhysRevC.84.024909.
  %[arXiv:1011.6053 [nucl-th]].
  %%CITATION = doi:10.1103/PhysRevC.84.024909;%%
  %74 citations counted in INSPIRE as of 30 Nov 2018  

%\bibitem{Gelis:2013rba} 
  T.~Epelbaum and F.~Gelis,
  %``Pressure isotropization in high energy heavy ion collisions,''
  Phys.\ Rev.\ Lett.\  {\bf 111}, 232301 (2013).

%\bibitem{Dusling:2010rm} 
  K.~Dusling, T.~Epelbaum, F.~Gelis and R.~Venugopalan,
  %``Role of quantum fluctuations in a system with strong fields: Onset of hydrodynamical flow,''
  Nucl.\ Phys.\ A {\bf 850}, 69 (2011).

%\bibitem{Vredevoogd:2008id} 
  J.~Vredevoogd and S.~Pratt,
  %``Universal Flow in the First Stage of Relativistic Heavy Ion Collisions,''
  Phys.\ Rev.\ C {\bf 79}, 044915 (2009).
 % doi:10.1103/PhysRevC.79.044915.
 % [arXiv:0810.4325 [nucl-th]].
  %%CITATION = doi:10.1103/PhysRevC.79.044915;%%
  %80 citations counted in INSPIRE as of 02 Dec 2018

\end{comment}
         
\end{thebibliography}


\end{document}


